\section{Introduction}



%%%%%%%%%%%%%%%%%%%%%%%%%%%%%%%%%%%%%%%%%
\begin{frame}[fragile]{But ... What is poetry?}

\begin{block}{Definition}
{\small \textbf{Poetry} is a piece of writing or speaking, which
\textbf{\textcolor{red}{MUST}} follow specific
\alert{\underline{\textbf{Patterns}}}.
}\end{block}

%People can easily determine whether a piece of writing is a poem or prose, but ONLY specialists can determine the class of poem.\\

\begin{example}
	\begin{center}
		\begin{Arabic}
			%@@@change this Bayt
			وُلِـدَ الـهُـدى فَـالكائِناتُ ضِياءُ\hspace{3em}  وَفَـمُ الـزَمـانِ تَـبَـسُّـمٌ وَثَناءُ
		\end{Arabic}
	\end{center}
\end{example}

\end{frame}

%%%%%%%%%%%%%%%%%%%%%%%%%%%%%%%%%%%%%%%%%
\begin{frame}[fragile]{Arabic Prosody \textarabic{العَرُوض}}
\begin{columns}
\begin{column}{0.6\textwidth}
	\textit{Al-Farahidi} (718 – 786 CE) analyzed the Arabic poetry, then he discovered the \alert{\underline{\textbf{Patterns}}} which is the succession of consonants and vowels.
\end{column}
\begin{column}{0.4\textwidth}
		\begin{figure}
			\begin{center}
				\includegraphics[width=0.6\textwidth]{Figures/Al-Farahidi.jpg}
				\caption{\textit{Al-Farahidi}\\
{\tiny figure taken from \url{https://goo.gl/ZJySa8}}.}
			\end{center}
		\end{figure}
\end{column}
\end{columns}
\end{frame}

%%%%%%%%%%%%%%%%%%%%%%%%%%%%%%%%%%%%%%%%%
\begin{frame}[fragile]{Arabic Prosody \textarabic{العَرُوض}}

\begin{itemize}
	\item<1-> A \textbf{poem } is a set of verses having the same meter and rhyme.
	\item<2-> \textbf{Vowels}  carry one of \textarabic{◌َ  ◌ُ  ◌ِ} which represent the following short vowels /a/, /u/, /i/.
	\item<3-> \textbf{Consonants} carry \textarabic{◌ْ}.
	\item<4-> \textbf{Shadaa} indicates the letter is doubled \textarabic{◌ّ}. 
	\item<5-> \textbf{Tanween} \textit{harakah} and \textit{Noon} letter with consonant to the end of the word. It sounds /n/.
\end{itemize}

\end{frame}
%%%%%%%%%%%%%%%%%%%%%%%%%%%%%%%%%%%%%%%%%%%%%%


%%%%%%%%%%%%%%%%%%%%%%%%%%%%%%%%%%%%%%%%%
\begin{frame}[fragile]{Arabic Prosody \textarabic{العَرُوض}}
\begin{columns}
	
	\begin{column}{.7\textwidth} % Left column and width
		{\small 	
		- A \textbf{foot}(\textit{tafa'ilah} \textarabic{التفعيلة}) : is an \textbf{ordered} sequence of vowels and consonants. \\
		- \textbf{Meter} \textarabic{البحر}: is an \textbf{ordered} sequence of \alert{feet}. }

		\begin{table}
		\small
				\begin{tabular}[h!]{|c|c|} 
			\hline
			\textbf{Meter Name} & \textbf{Meter} \small{\textit{feet combination}} \\ 
			\hline
			\textit{al-Wafeer}    & \textarabic{مُفَاعَلَتُن مُفَاعَلَتُن فَعُولُن} \\ %
			\textit{al-Taweel}    & \textarabic{فَعُوْلُنْ مَفَاْعِيْلُنْ فَعُوْلُنْ مَفَاْعِلُنْ} \\ %
			\vdots                &  \vdots\\
			\textit{al-Moktadib}  & \textarabic{مَفْعُوْلاتُ مُسْتَفْعِلُنْ مُسْتَفْعِلُن} \\
			\textit{al-Modar'e}   & \textarabic{مَفَاْعِيْلُنْ فَاْعِلاتُنْ مَفَاْعِيْلُنْ} \\
			\hline
		\end{tabular}
		\end{table}
	\end{column}%
		\hfill%
		\begin{column}{.3 \textwidth}
			\begin{table}
				\small
			\begin{tabular}{|c|c|} \hline
			\textbf{Feet} & \textbf{Scansion} \\
			\hline
			\textarabic{فَعُولُنْ}  & \texttt{0/0//}\\
			\textarabic{فَاعِلُنْ}  & \texttt{0//0/}\\
			\textarabic{مُسْتَفْعِلُنْ}& \texttt{0//0/0/}\\
			\textarabic{مَفاعِيلُنْ}& \texttt{0/0/0//}\\
			\textarabic{مَفْعُولاَت} & \texttt{0//0///}\\
			\textarabic{فَاعِلاَتُنْ} & \texttt{0/0//0/}\\
			\textarabic{مُفَاعَلَتُنْ}& \texttt{0///0//}\\
			\textarabic{مُتَفَاعِلُنْ}& \texttt{0//0///}\\
			\hline
		\end{tabular}
		\end{table}		

	\end{column}
\end{columns}
\begin{center}
	\begin{Arabic}
		\begin{table}
			\small
			\begin{tabular}[h!]{c c c c c c} 
%				\hline
				\multicolumn{3}{c} \hfil \textarabic{ وَمَـنْ دَعَــا النَّــاسَ إِلَى ذَمِّــهِ}  \hfil  & \multicolumn{3}{c} \hfil \textarabic{ذَمُّــوْهُ بِالْحَــقِّ وبِالْبَــاطِلِ} \hfil \\ 
				{\color{purple} \textarabic{وَمَنْ دَعَ نـْ}} & {\color{blue} \textarabic{ ـنَاسَ إِلَى}}& {\color{OliveGreen}\textarabic{ ذَمْمِهِى}} & 
				{\color{purple}\textarabic{ ذَمْمُوْهُ بِلـْ}} &{\color{blue} \textarabic{ ـحَقْقِ وبِلـْ}} &{\color{OliveGreen}\textarabic{ ـبَاطِلِيْ}}\\
				{\color{purple} \texttt{0//0//}} & {\color{blue} \texttt{0///0/}} & {\color{OliveGreen} \texttt{0//0/}} &
				{\color{purple} \texttt{0//0/0/}} & {\color{blue} \texttt{ 0///0/}} & {\color{OliveGreen} \texttt{0//0/}}\\
				{\color{purple} \textarabic{مُتَفْعِلُنْ}} &{\color{blue}\textarabic{ مُسْتَعِلُنْ}} &{\color{OliveGreen} \textarabic{مَفْعُلا}}&
				{\color{purple}\textarabic{مُسْتَفْعِلُنْ}}& {\color{blue}\textarabic{ مُسْتَعِلُنْ}}& {\color{OliveGreen}\textarabic{ مَفْعُلا}}\\
%				\hline
			\end{tabular}
		\end{table}
	\end{Arabic}%
	
\end{center}


\end{frame}

%%%%%%%%%%%%%%%%%%%%%%%%%%%%%%%%%%%%%%%%%%%%%%%%%%%%%%%%%%%%%

%%%%%%%%%%%%%%%%%%%%%%%%%%%%%%%%%%%%%%%%%
\begin{frame}
\frametitle{Thesis Working Steps.}
\begin{figure}[!t]
	\input{./Figures/Fig_Master_Cycle.tex}
	\caption{Thesis Working Steps.}
	\label{Fig:Thesis_Cycle}
\end{figure}

\end{frame}
%%%%%%%%%%%%%%%%%%%%%%%%%%%%%%%%%%%%%%%%%
%
%\begin{frame}
%\frametitle{Sample frame title}
%This is a text in second frame. For the sake of showing an example.
%
%\begin{itemize}
%    \item<1-> Text visible on slide 1
%    \item<2-> Text visible on slide 2
%    \item<3-> Text visible on slide 3
%    \item<4-> Text visible on slide 4
%\end{itemize}
%\end{frame}
%
%%%---------------------------------------------------------
%%%Example of the \pause command
%\begin{frame}
%In this slide \pause
%
%the text will be partially visible \pause
%
%And finally everything will be there
%\end{frame}
%
%\section{Second section}
%
%%%---------------------------------------------------------
%%%Highlighting text
%\begin{frame}
%\frametitle{Sample frame title}
%
%In this slide, some important text will be
%\alert{highlighted} beause it's important.
%Please, don't abuse it.
%
%\begin{block}{Remark}
%Sample text
%\end{block}
%
%\begin{alertblock}{Important theorem}
%Sample text in red box
%\end{alertblock}
%

%\end{frame}
%%%---------------------------------------------------------
%%
%%
%%%---------------------------------------------------------
%%%Two columns
%\begin{frame}
%\frametitle{Two-column slide}
%
%\begin{columns}
%
%\column{0.5\textwidth}
%This is a text in first column.
%$$E=mc^2$$
%\begin{itemize}
%\item First item
%\item Second item
%\end{itemize}
%
%\column{0.5\textwidth}
%This text will be in the second column
%and on a second tought this is a nice looking
%layout in some cases.
%\end{columns}
%\end{frame}
%%---------------------------------------------------------



%%%%%%%%%%%%%%%%%%%%%%%%%%%%%%%%%%%%%%%%%%%%%%%%%%%%%%%%%%%%%%%%%%%%%%%%%%%
%%% Local Variables:
%%% mode: latex
%%% TeX-master: "../main"
%%% TeX-engine: xetex
%%% End:
