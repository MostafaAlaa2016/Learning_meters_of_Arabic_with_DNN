\section{Literature Review}

\begin{frame}[fragile]{Literature Review:Detecting Arabic poems meters}

\begin{block}{Deterministic Approach}
	There is some literature on recognizing the meters of written Arabic poem using rule-based deterministic algorithms
\end{block}

\begin{itemize}
	\item <1-> [--] \textbf{Abuata and Al-Omari~\cite{Abuata2016RuleBasedAlgorithm}}:
	\begin{itemize}
		\item<2-> Deterministic Algorithm
		\begin{enumerate}
			\item Getting the input, carrying full diacritics.
			\item Metrical scansion rules are applied to the Arud writing. 0/0/..
			\item Grouping zero and ones to feet \textarabic{تفعيلات}.
			\item A class is assigned  to the input.
		\end{enumerate}
		\item<3-> \textbf{\alert{Results}}: 82.2\% of 417 verses.
	\end{itemize}
	\item <4->[--]\textbf{Alnagdawi et al~\cite{Alnagdawi2013FindingArabicPoemMeter}}, similar approach;  Context-Free Grammar; 75\% correctly
	classed from 128.
\end{itemize}


\end{frame}

\begin{frame}[fragile]{Machine Learning approach: Our point of departure}
\begin{itemize}
\item[--]<1-> Dataset issues:
\begin{itemize}
\item<2-> Dataset size.
\item<3-> Diacritics are a must.
\item<4-> Converting verses into Al-Arud writing style is probabilistic.
\end{itemize}

\item[--]<5-> Detection issues:

	\begin{itemize}
	\item<6-> For detecting meters, all models are so \textbf{naive and primitive}. They do no have any clue about the real pattern.
	\item<7-> Accuracies: (75\%, 82\%) tested on (128, 417) verses respectively.
	
	% A source of randomness.
	%s. Treating such a problem as a deterministic problem will not satisfy the case study. It results in many limitations, including obligating verses to have full diacritics on every single letter before conducting the classification
	
	\item<8-> Encoding technique.
	\end{itemize}
\end{itemize}
\end{frame}
%%%%%%%%%%%%%%%%%%%%%%%%%%%%%%%%%%%%%%%%%%%%%%%%%

%%%%%%%%%%%%%%%%%%%%%%%%%%%%%%%%%%%%%%%%%%%%%%%%%
%%%%%%%%%%%%%%%%%%%%%%%%%%%%%%%%%%%%%%%%%%%%%%%%%%%%%%%%%%%%%%%%%%%%%%%%%%%
%%% Local Variables:
%%% mode: latex
%%% TeX-master: "../main"
%%% TeX-engine: xetex
%%% End:
