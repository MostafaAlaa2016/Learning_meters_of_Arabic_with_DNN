\begin{tikzpicture}[%
  shorten >=1pt,->,draw=black!50, node distance=2.5cm,
  %every pin edge={<-,shorten <=1pt},
  neuron/.style={circle,fill=black!25,minimum size=17pt,inner sep=0pt},
  input neuron/.style={neuron, fill=green!50},
  output neuron/.style={neuron, fill=red!50},
  hidden neuron/.style={neuron, fill=blue!50},
  annot/.style = {text width=4em, text centered}
  ]
  \def\layersep{2.5cm}
  % Draw the input layer nodes
  \foreach \name / \y in {1,...,4}
  % This is the same as writing \foreach \name / \y in {1/1,2/2,3/3,4/4}
  \node[input neuron, pin={[pin edge={<-,shorten <=1pt}]left:Input \#\y}] (I-\name) at (0,-\y) {};

  % Draw the hidden layer nodes
  \foreach \name / \y in {1,...,5}
  \node[hidden neuron] (H-\name) at (\layersep,-\y cm+0.5cm) {};

  % Draw the output layer node
  \node[output neuron,pin={[pin edge={->}]right:Output}, right=of H-3] (O) {};

  % Connect every node in the input layer with every node in the
  % hidden layer.
  \foreach \source in {1,...,4}
  \foreach \dest in {1,...,5}
  \path (I-\source) edge (H-\dest);

  % Connect every node in the hidden layer with the output layer
  \foreach \source in {1,...,5}
  \path (H-\source) edge (O);

  % Annotate the layers
  \node[annot,above=15mm of H-1] (hl) {Hidden layer};
  \node[annot,left=of hl] {Input layer};
  \node[annot,right=of hl] {Output layer};
  %%% 
  \coordinate (H-I-1) at ($(H-1)!0.5!(I-1)$);
  \draw (O) |- ($(H-I-1) +(0,1)$) node[above,pos=0.75]{Back propagation} -- (H-I-1);
  \draw ;
\end{tikzpicture}