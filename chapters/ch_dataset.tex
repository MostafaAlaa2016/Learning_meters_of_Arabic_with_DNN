\chapter{\uppercase{Dataset: Acquisition and Encoding}}\label{Ch:Datasets}  

In this chapter, We introduces the Dataset acquisition and encoding including the essential pre-processing steps, and the justification for their need. Pre-processing steps are data extraction, data cleansing, data format, and data encoding techniques used. Also, it contains comparisons between the three techniques used.

  The collection of the dataset was one of the most laborious tasks in this project. There were criteria we were searching to find. These criteria are as follows,
  \begin{itemize}
    
  \item \textbf{Datasets availability:} There are old Arabic references which have a lot of Poems but not all these books were not available in a PDF or a Web pages format, and it was hard to find it.
    
  \item \textbf{The Poem with diacritics:} There are resources which have Arabic Poems, but it is much harder to find same with diacritics.
    
  \item \textbf{The amount of the dataset:} To have a successful project with good results we need a massive amount of data. From the previous work, We did not find this amount of data. The maximum number found was 1.5k. However, We were searching for around 1.5M record of classified poetry.

  
\item \textbf{Cleansing of this data:} There was a limitation for the datasets which we can consider it, or we can scrap it due to the limitation for the APIs or the ready datasets in this context.
  
\end{itemize}
To meet the above criteria and overcome it, We applied following,

\begin{itemize}

  \item \textbf{Datasets availability:} We have scrapped the Arabic datasets from two big poetry websites: \textarabic{الديوان}~\cite{diwan}, \textarabic{الموسوعة الشعرية}~\cite{PoetryEncyclopedia2016}. Both merged into one large dataset, and we open sourced it online ~\cite{ArabicpoetryDS}.

  \item \textbf{The Poem with diacritics:} We tried to get the most verses with the available diacritics, but the diacritics states are not consistent, So, a verse can be fully diacritics, Semi diacritics or without diacritics.

\item \textbf{The amount of the dataset:} The total number of verses is 1,862,046 poetic verses; each verse labeled by its meter (class), the poet who wrote it, and the age which it was written. There are 22 meters, 3701 poets and 11 ages; and they are Pre Islamic, Islamic, Umayyad, Mamluk, Abbasid, Ayyubid, Ottoman, Andalusian, the era between Umayyad and Abbasid, Fatimid and modern. We are only interested in the 16 classic meters which attributed to Al-Farahidi, and they are the majority of the dataset with a total number of 1,722,321 verses. Figure~\ref{Fig:Data_Size_Distribution} shows the distribution of the verses per meter. %@@@ add datasets figures percentage per class
  
\item \textbf{Cleansing of this data:} Dataset was not cleaned enough for usage in this research, but we have applied cleansing rules explained in details in Data Preparation and Cleansing section~\ref{sec:Data_Clens}. We also open sourced all the code scripts used in our online repository~\cite{HCILAB_ArabicPoetry_2018}.
\end{itemize}

\begin{figure}[!t]
	\centering
	\begin{tikzpicture}
	\begin{axis}[
    symbolic x coords={Taweel,
   Kamel,
   Baseet,
   Khafeef,
   Wafeer,
   Rigz,
   Raml,
   Motakarib,
   Sar'e,
   Monsareh,
   Mogtath,
   Madeed,
   Hazg,
   Motadarik,
   Moktadib,
   Modar'e
    },
    xtick=data,
    % the following x label positioning does work here.
    every axis y label/.style= {at={( 0.15, 1.07)}, anchor=north},
    ylabel style={font=\footnotesize},
    xticklabel style = {font=\footnotesize},
    ylabel={Verses},
    height=8cm,
    x=0.4cm,
    x tick label style={rotate=90, anchor=east},
    enlarge y limits=0.07,
    name=left plot,%
    title=(a) \textit{Arabic Dataset},%
    title style={at={(0.5,-.4)}}%
  ]
    \addplot[ybar,color=black,mark=*, only marks,
	    point meta=explicit symbolic] coordinates {
        (Taweel, 416428)
        (Kamel,  370116)
        (Baseet, 244583)
        (Khafeef,     157880)
        (Wafeer,     143148)
        (Rigz,     119286)
        (Raml,     79560)
        (Motakarib,     63613)
        (Sar'e,     59370)
        (Monsareh,     28768)
        (Mogtath,     18062)
        (Madeed,     7808)
        (Hazg,     7468)
        (Motadarik,     5144)
        (Moktadib,     799 )
        (Modar'e,     288 )
    };

    \draw[loosely dotted] (axis cs:Taweel, 0) -- (axis cs:Taweel, 416428);
    \draw[loosely dotted] (axis cs:Kamel,  0) -- (axis cs:Kamel,  370116);
    \draw[loosely dotted] (axis cs:Baseet, 0) -- (axis cs:Baseet, 244583);
    \draw[loosely dotted] (axis cs:Khafeef,0) -- (axis cs:Khafeef,157880);
    \draw[loosely dotted] (axis cs:Wafeer, 0) -- (axis cs:Wafeer, 143148);
    \draw[loosely dotted] (axis cs:Rigz,   0) -- (axis cs:Rigz,   119286);
    \draw[loosely dotted] (axis cs:Raml,   0) -- (axis cs:Raml,   79560);
    \draw[loosely dotted] (axis cs:Motakarib, 0) -- (axis cs:Motakarib, 63613);
    \draw[loosely dotted] (axis cs:Sar'e, 0)   -- (axis cs:Sar'e, 59370);
    \draw[loosely dotted] (axis cs:Monsareh, 0) -- (axis cs:Monsareh,28768);
    \draw[loosely dotted] (axis cs:Mogtath, 0) -- (axis cs:Mogtath, 18062);
    \draw[loosely dotted] (axis cs:Madeed,  0) -- (axis cs:Madeed,  7808);
    \draw[loosely dotted] (axis cs:Hazg,    0) -- (axis cs:Hazg,    7468);
    \draw[loosely dotted] (axis cs:Motadarik,0) -- (axis cs:Motadarik, 5144);
    \draw[loosely dotted] (axis cs:Moktadib, 0) -- (axis cs:Moktadib,  799 );
    \draw[loosely dotted] (axis cs:Modar'e,  0) -- (axis cs:Modar'e,   288 );

\end{axis}



	\end{tikzpicture}%
	\caption{Arabic dataset Meter per class percentage ordered descendingly on x axis vs. corresponding meter name on y axis all class in the left of the red line (less than 1\% assume to be trimmed in some experiments).	}\label{Fig:Data_Size_Distribution}
\end{figure}

\section{Data Scraping}\label{sec:Data_Scrap}
To scrap the data from the website: \textarabic{الديوان}~\cite{diwan},ends up into such a problem just reduce your problem to the most smallest one. That means: First: Check if any "keywords" is set, if used. Then: Use your whole preamble and print the complete bibliography. If this ends up in the same error, your problem might be in your preamble. -> Reduction of preamble, until you get your bib printed. Adding slowly parts back to the preamble, until the error occurs again. That might show you, what lead to the warning.

 \textarabic{الموسوعة الشعرية}~\cite{PoetryEncyclopedia2016}, We used custom Python scripts for each websites to get the verses details. The script created with simple usage to pass the link we need to scrap. We will show two examples from both websites.
\begin{enumerate}
\item The First example, If we need to scrap a meter from \textarabic{الديوان} the website, for example Al-Tawil \\\url{https://www.aldiwan.net/poem.html?Word=\%C7\%E1\%D8\%E6\%ED\%E1\&Find=meaning}, We will pass this link to the script and the output file name. The script will start scraping and save the output in a CSV format. We can get the output similar than the output in table \ref{Tab:Aldiwan_Sample}
\item Second Example, If we need to scrap the same meter from \textarabic{الموسوعة الشعرية} the website for example Al-Raml \url{https://poetry.dctabudhabi.ae/\#/diwan/poem/126971}, We will pass this link to the script and the output file name. The script will start scraping and save the output in a CSV format. We can get the output similar than the output in table \ref{Tab:ElMosoaa_Sample}
We scrapped all the available datasets on both websites and merged them based on the common columns. Then we started the Data preparation tasks. We need to mention that, Not all diacritics was correctly available on all the websites. Also, We did not work to generate the diacritics for those datasets. So, we depended on whatever available without changing the data all the next sections is related to correction, preparation, and cleansing of the current datasets.

\section{Data Preparation and Cleansing}\label{sec:Data_Clens}

Data preparation and cleansing tasks divided into multi-stages.

\begin{itemize}
\item Merge all scrapped datasets into one CSV file with a selection of the common columns in each file.
\item Remove the duplicates rows from the files in case we have any joined rows between both websites.
\item Filter the datasets on the 16 meters required as some data belonged to other non-famous or not original meters.
\item Remove many unnecessary white spaces which were useless.
\item Remove non-Arabic characters and the other web symbols.
\item Fix diacritics mistakes, such as the existence of two consecutive harakat, we have only kept one and have removed the other. %@@@add an example
\item Remove any \textit{harakah} comes after a white space, it removed as it is useless. %@@@add an example
\item We factored \textit{Shadaa}~\ref{def:shadaa_definition} to its original format explained in this example~\ref{Tab:Shadda_Dal} previously.
\item We also factored \textit{Tanween}~\ref{def:tanween_definition} to its original format explained in this example~\ref{Tab:Tanween_Dal} previously.\footnote{\textit{We ignored the factorization of Alef-Mad  \textbf{\textarabic{ آ }} in our data preparation and transformation which can save more memory and shorten our encoding vectors}}
\end{itemize}

We need to highlight that the last two points are not a handcrafted feature. It is a factorization for the letter to its original format. This factorization will affect the size of the data in the memory and the letter representation in the vector. We will explain this part in details in the next chapter about encoding mechanism and the impact of the encoding type in the model training time and performance.

%\clearpage
% table: dal with diacritics
\begin{table}[!t]
	\centering
	\begin{tabular*}{\textwidth}{c @{\extracolsep{\fill}}c c c c c}
		%\hline
		\toprule
          \textbf{\small{\textarabic{البيت}}} & \small{\textbf{\textarabic{الشطر الأيسر}}} & \small{\textbf{\textarabic{الشطر الأيمن}}} &
\small{\textbf{\textarabic{البحر}}} & \small{\textbf{\textarabic{الشاعر}}} \\
		 %\hline
          \midrule
\makecell{\textarabic{رَجا شافع نسج المودّة بيننا}\\ \textarabic{ولا خيرَ في ودّ يكون بشافع}} &
\textarabic{ولا خيرَ في ودّ يكون بشافع} &                                                       \textarabic{رَجا شافع نسج المودّة بيننا} &                                                       \textarabic{الطويل}&
\textarabic{ابن نباته المصري}\\
          
		%\hline
		\bottomrule
	\end{tabular*}
	\caption{Aldiwan scraping output example }\label{Tab:Aldiwan_Sample}
\end{table}


% table: dal with diacritics
\begin{table}[!t]
	\centering
	\begin{tabular*}{\textwidth}{c @{\extracolsep{\fill}}c c c c c c c c c}
		%\hline
          \toprule
\small{\textbf{\#}} &
\small{\textbf{\textarabic{البيت}}} &
\small{\textbf{\textarabic{الشطر الأيمن}}}&                        \small{\textbf{\textarabic{الشطر الأيسر}}} &
\small{\textbf{\textarabic{البحر}}}&                                 \small{\textbf{\textarabic{القافية}}}& \small{\textbf{\textarabic{الديوان}}}&                               \small{\textbf{\textarabic{الشاعر}}}&
\small{\textbf{\textarabic{العصر}}}\\
		 %\hline
          \midrule
1 &          
\makecell{\textarabic{من يرد مورد حب} \\ \textarabic{ظمأ بالشوق يزدد}} &
\textarabic{ظمأ بالشوق يزدد} &                                                        \textarabic{من يرد مورد حب} &                                                       \textarabic{الرمل}&
\textarabic{د}&
\makecell{\textarabic{الديوان} \\ \textarabic{الرئيسي}}&
\makecell{\textarabic{يعقوب الحاج}\\ \textarabic{ جعفر التبريزي}}&
\textarabic{الحديث}\\
          
		%\hline
		\bottomrule
	\end{tabular*}
	\caption{Al-Mosoaa Elshearyaa scraping output example }\label{Tab:ElMosoaa_Sample}
\end{table}
\end{enumerate}


\section{\uppercase{Data Encoding}}\label{Ch:Data_Encoding}

As we explained, We have collected the dataset and cleaned the data from any quality issues. The next step is to change the data representation to be ready for model training. This change of the data structure named \textit{Data Encoding}.

\subsection{Encoding in English}


\begin{itemize}
\item \textbf{Work embedding Encoding in English} The concept of data encoding was first introduced by [Bengio et al., 2003]~\cite{Bengio2003}. They used an embedding lookup table as a reference and map every word to this lookup. They used the resulting dense vectors as input for language modeling. There are many works to improve the word embedding one of them [Collobert et al., 2011]~\cite{Collobert_2011} proposed improvement of word embedding task and proved the versatility of word embedding in many NLP tasks. Another work proposed by [Mikolov et al., 2013~\cite{Mikolov_2013};
Jeffrey Pennington et al., 2014~\cite{Pennington_2014} ] shows the maturity of word embedding and is currently the most used encoding technique in the neural network based natural language processing.

\item \textbf{Character Level Encoding in English} 
All the previous work focused on word embedding encoding, but in our research problem here we do not work on word level we focus into character level encoding as input feature to the model. There is a good deal of research based on the character level encoding [Kim et al., 2015]~\cite{Kim_2015} used character level embedding to construct word level representations to work on out of vocabulary problem. [Chiu and Nichols, 2015]~\cite{Chiu_2015} also used character embeddings with a convolutional neural network for named entity recognition.[Lee, Jinhyuk et al.,2017]~\cite{ijcai_2017} used character embeddings for the personal name classification using Recurrent Neural Networks.

\end{itemize}

\subsection{Character Level Encoding in Arabic}\label{sec:Char_Level_Arabic}

Working on Arabic language embedding based on the character level did not take much attention from the research community. [Potdar et al.,2017]~\cite{Potdar_2017} has done a comparative study on six encoding techniques. We are interested in the comparison of one-hot and binary. They have used Artificial Neural Network for evaluating cars based on seven ordered qualitative features. The accuracy of the model was the same in both encoding one-hot and binary. [Agirrezabal et al.,2017]\cite{Agirrezabal_2017} shows that representations of data learned from character-based neural models are more informative than the ones from hand-crafted features.

In this research, We will make a comparative study of different encoding techniques between binary and one-hot. Also, we provide some new encoding method specific for Arabic letters, and we will see the effect of this on our problem. We will show the efficiency of every technique based on performing model training and model running time performance.

Generally, a character will be represented as an n vector. Consequently, a verse would be an $n \times p$ matrix, where n
is the character representation length and p is the verse’s length, n varies from one encoding to another, we have used One-Hot and Binary encoding techniques and proposed a new encoding, the \textbf{Two-Hot} encoding.

Arabic letters have a feature related to the diacritics; To explain this feature we will take an example based on \textit{One-Hot} encoding. This feature is related to how we will treat the character with a diacritic. Arabic letters are 36 + white space as a letter. So, the total is 37. Any letter represented as a vector $37 \times 1$. Let's take an example a work such as \textarabic{مرحبا} having 5 letters encoded as a $37 \times 5$ matrix. If it came with diacritics such as \textarabic{مَرْحَبَا} and we need to represent the letters as One-Hot encoding we will consider every letter and diacritics as a separate letter. So, it will be 5 character and 4 diacritics. The vector shape will be $41 \times 9$.

One of the main reason we need to care about the encoding is the  \textit{RNN} training. If we have a different number of time steps in  \textit{RNN} cell and the input vector dimensions are different based on the input, It will have a standard architecture for the model and to be able to train both the work with diacritics and without diacritics to show the effect of the model learning on the same architecture.

To achieve the model architecture unification,  we proposed three different encoding systems: \textit{one-hot}, \textit{binary}, and the novel encoding system developed in this project \textit{two-hot}. The three of them explained in the next three subsections.

\begin{figure*}[!t]
  \centering
    \begin{tikzpicture}
    \node at (-12,0){\begin{tikzpicture}   
  % Rectangles
  % the space between each rectangular
  \def \x {0.57} 
  \def \s {1} % I'll use it to shift to right
  % A list of letters <3
  \newcommand\letters{{"ا","بَ","حَ", "رْ", "مَ"}}

  % Rectangle
  \def \length {2.95}
  \foreach \j in {0,...,4} 
  \draw [rounded corners=1pt] (\s+\x*\j, 0) rectangle (\s+0.5+\x*\j, \length) 
  node[above, xshift=-0.25cm]
  {\textarabic{\pgfmathparse{\letters[\j]}\pgfmathresult}};

  % Above Annotation
  % Cell line
  % 0.25*10+0.09 = 2.59
  \def \d {3.9 - 0.15}
  \draw (\s+0.25-0.007, 3.9 ) -- (\s+2.53+0.007, 3.9 ); % the main bar
  \draw (\s+0.25, 3.9 ) -- (\s+0.25, \d ); % sub bar #ا
  \draw (\s+0.25*3+0.03, 3.9 ) -- (\s+0.25*3+0.03, \d ); % sub bar #ب
  \draw (\s+0.25*5+0.15, 3.9+0.15 ) -- (\s+0.25*5+0.15, \d ) 
  node[above, yshift=0.3cm]{\textarabic{مَرْحَبَا}}; % sub bar #ح
  \draw (\s+0.25*7+0.2, 3.9 ) -- (\s+0.25*7+0.2, \d ); % sub bar #ر
  \draw (\s+2.53, 3.9 )  -- (\s+2.53, \d );  % sub-bar #م

  % adding numbers, vector ا
  \node at (\s+0.25, \length - 0.25)    {1}; % the top number; using -
  \node at (\s+0.25, \length - 0.6)     {0};
  \node at (\s+0.25, \length - 0.95)    {0};
  \node at (\s+0.25, \length - 1.3)     {0};
  \node at (\s+0.25,  1.1)     {\vdots};
  % \node at (\s+0.25, \length - 1.65)    {0};
  % \node at (\s+0.25, \length - 2)       {0};
  % \node at (\s+0.25, \length - 2.35)    {0};
  \node at (\s+0.25, \length - 2.7)     {0};

  % distance between vectors elements
  \def \d {0.07}
  % adding numbers, vector ب
  \node at (\s+0.25*3+\d, \length - 0.25)    {0}; % the top number; using -
  \node at (\s+0.25*3+\d, \length - 0.6)     {\vdots};
  % \node at (\s+0.25*3+\d, \length - 0.95)    {0};
  \node at (\s+0.25*3+\d, \length - 1.18)     {1};
  \node at (\s+0.25*3+\d, 1.28) {\vdots};
  \node at (\s+0.25*3+\d, \length - 2.35)    {0};
  \node at (\s+0.25*3+\d, 0.25) {0}; % The bottom  number


  % adding numbers, vector ح
  \node at (\s+0.25*5+\d*2, \length - 0.25)    {0}; % the top number; using -
  \node at (\s+0.25*5+\d*2, \length - 0.6)     {\vdots};
  \node at (\s+0.25*5+\d*2, \length - 1.35)     {1};
  \node at (\s+0.25*5+\d*2, 1.18) {\vdots};
  \node at (\s+0.25*5+\d*2, \length - 2.35)    {0};
  \node at (\s+0.25*5+\d*2, 0.25) {0}; % The bottom  number

  % adding numbers, vector ر
  \node at (\s+0.25*7+\d*3, \length - 0.25)    {0}; % the top number; using -
  \node at (\s+0.25*7+\d*3, \length - 0.6)     {0};
  \node at (\s+0.25*7+\d*3, \length - 0.95)     {0};
  \node at (\s+0.25*7+\d*3, \length - 1.33)     {\vdots};
  \node at (\s+0.25*7+\d*3, 1.07) {1};
  \node at (\s+0.25*7+\d*3, \length - 2.2)    {\vdots};
  \node at (\s+0.25*7+\d*3, 0.25) {0}; % The bottom  number

  % adding numbers, vector م
  \node at (\s+0.25*9+\d*4, 3 - 0.25) {0}; % the top number; using -
  \node at (\s+0.25*9+\d*4, 3 - 0.69) {0};
  \node at (\s+0.25*9+\d*4, 0.25) {0};
  \node at (\s+0.25*9+\d*4, 0.25*4 - 0.1) {\vdots};
  \node at (\s+0.25*9+\d*4, 0.9 + 0.4) {1};
  \node at (\s+0.25*9+\d*4, 1.3 + 0.6) {\vdots};

  % Vector size annotations.
  % figures starts at x = 1
  \def \d {0.6} % Annotation x=0.4
  \draw[arrows=->] (\d, 1.7) -- (\d, 2.9); % up-line
  \draw[arrows=<-] (\d, 0.1) -- (\d, 1.1) node[above, yshift=0.07cm]
  {\small{181}};
  % bottom-line

  % \draw[arrows=<-] (\s+0.1, -0.4) -- (\s+1.19, -0.4) node[right, xshift=0.05cm] {\small{5}};
  % \draw[arrows=->] (\s+1.2+0.5, -0.4) -- ( \s+2.7,-0.4);
\end{tikzpicture}%

%%% Local Variables:
%%% mode: latex
%%% TeX-master: "../main"
%%% End:

};
    \node at (-7,0){\begin{tikzpicture}[x=1cm,y=1cm]

  % Rectangles
  % the space between each rectangular
  \def \x {0.57} 
  \def \s {1} % I'll use it to shift to right
  % A list of letters <3
  \newcommand\letters{{"ا","بَ","حَ", "رْ", "مَ"}}

  % \draw[step=0.2, gray, very thin, red] (1,-0.5) grid (4,4);

  % Rectangulars
  \def \length {2.95}
  \foreach \j in {0,...,4} 
  \draw[rounded corners=1pt] (\s+\x*\j, 0) rectangle (\s+0.5+\x*\j, \length) 
  node[above, xshift=-0.25cm]
  {\textarabic{\pgfmathparse{\letters[\j]}\pgfmathresult}};

  % Above Annotatios
  % Cell line
  % 0.25*10+0.09 = 2.59
  \def \d {3.9 - 0.15}
  \def \z {0.2}
  \draw (\s+0.25-0.007, 3.9 ) -- (\s+2.53+0.007, 3.9 ); % the main bar
  \draw (\s+0.25, 3.9 ) -- (\s+0.25, \d ); % sub bar #ا
  \draw (\s+0.25*3+0.03, 3.9 ) -- (\s+0.25*3+0.03, \d ); % sub bar #ب
  \draw (\s+0.25*5+0.15, 3.9+0.15 ) -- (\s+0.25*5+0.15, \d ) 
  node[above, yshift=0.3cm]{\textarabic{مَرْحَبَا}}; % sub bar #ح
  \draw (\s+0.25*7+0.2, 3.9 ) -- (\s+0.25*7+0.2, \d ); % sub bar #ر
  \draw (\s+2.53, 3.9 )  -- (\s+2.53, \d );  % sub-bar #م

  % adding numbers, vector ا
  \node at (\s+0.25, \length - 0.25)    {0}; % the top number; using -
  \node at (\s+0.25, \length - 0.6)     {0};
  \node at (\s+0.25, \length - 0.95)    {0};
  \node at (\s+0.25, \length - 1.3)     {0};
  \node at (\s+0.25, \length - 1.65)    {0};
  \node at (\s+0.25, \length - 2)       {0};
  \node at (\s+0.25, \length - 2.35)    {0};
  \node at (\s+0.25, \length - 2.7)     {1};

  % distance between vectors elements
  \def \d {0.07}
  % adding numbers, vector ب
  \node at (\s+0.25*3+\d*1, \length - 0.25)    {0}; % the top number; using -
  \node at (\s+0.25*3+\d*1, \length - 0.6)     {0};
  \node at (\s+0.25*3+\d*1, \length - 0.95)    {1};
  \node at (\s+0.25*3+\d*1, \length - 1.3)     {0};
  \node at (\s+0.25*3+\d*1, \length - 1.65)    {0};
  \node at (\s+0.25*3+\d*1, \length - 2)       {1};
  \node at (\s+0.25*3+\d*1, \length - 2.35)    {1};
  \node at (\s+0.25*3+\d*1, \length - 2.7)     {1};


  % adding numbers, vector ح
  \node at (\s+0.25*5+\d*2, \length - 0.25)    {0}; % the top number; using -
  \node at (\s+0.25*5+\d*2, \length - 0.6)     {0};
  \node at (\s+0.25*5+\d*2, \length - 0.95)    {1};
  \node at (\s+0.25*5+\d*2, \length - 1.3)     {0};
  \node at (\s+0.25*5+\d*2, \length - 1.65)    {1};
  \node at (\s+0.25*5+\d*2, \length - 2)       {0};
  \node at (\s+0.25*5+\d*2, \length - 2.35)    {1};
  \node at (\s+0.25*5+\d*2, \length - 2.7)     {1};

  % adding numbers, vector ر
  \node at (\s+0.25*7+\d*3, \length - 0.25)    {1}; % the top number; using -
  \node at (\s+0.25*7+\d*3, \length - 0.6)     {0};
  \node at (\s+0.25*7+\d*3, \length - 0.95)    {0};
  \node at (\s+0.25*7+\d*3, \length - 1.3)     {1};
  \node at (\s+0.25*7+\d*3, \length - 1.65)    {1};
  \node at (\s+0.25*7+\d*3, \length - 2)       {0};
  \node at (\s+0.25*7+\d*3, \length - 2.35)    {1};
  \node at (\s+0.25*7+\d*3, \length - 2.7)     {1};

  % adding numbers, vector م
  \node at (\s+0.25*9+\d*4, \length - 0.25)    {0}; % the top number; using -
  \node at (\s+0.25*9+\d*4, \length - 0.6)     {0};
  \node at (\s+0.25*9+\d*4, \length - 0.95)    {1};
  \node at (\s+0.25*9+\d*4, \length - 1.3)     {1};
  \node at (\s+0.25*9+\d*4, \length - 1.65)    {1};
  \node at (\s+0.25*9+\d*4, \length - 2)       {1};
  \node at (\s+0.25*9+\d*4, \length - 2.35)    {0};
  \node at (\s+0.25*9+\d*4, \length - 2.7)     {1};

  % Vector size annotations.
  % figures starts at x = 1
  \def \vColumn {0.6} % Annotation x=0.4
  \draw[arrows=->] (\vColumn, 1.7) -- (\vColumn, 2.9); % up-line
  \draw[arrows=<-] (\vColumn, 0.1) -- (\vColumn, 1.1) node[above, yshift=0.07cm] {\small{8}};
  % bottom-line

  % \draw[arrows=<-] (\s+0.1, -0.4) -- (\s+1.19, -0.4) node[right, xshift=0.05cm] {\small{5}};
  % \draw[arrows=->] (\s+1.2+0.5, -0.4) -- ( \s+2.7,-0.4);

\end{tikzpicture}


%%% Local Variables:
%%% mode: latex
%%% TeX-master: "../main"
%%% End:

};
    \node at (0,0){% Defining a new box
\newsavebox{\columnVector}
% Populating the box
\savebox{\columnVector}{$\left[\begin{smallmatrix}k\\m\end{smallmatrix}\right]$}

\begin{tikzpicture}

  % The grid
  % \draw[step=0.5, gray!40, very thin] (0,0) grid (6,4);

  \def \recWidthA {0.5}
  \def \yStartA   {1.5}
  \def \yEndA     {3}
  \def \xStartA   {1.5}
  \def \xEndA     {0.5}
  \def \xStartB   {0}
  \def \yStartB   {1}
  \def \xEndB     {2}
  \def \yEndB     {3.5}
  \def \half      {\yEndA/2 - \yStartA/2}
  \def \xStartC   {3 +0.5}
  \def \yStartC   {0.5 -.1}
  \def \xEndC     {3.5 +0.5}
  \def \yEndC     {4 +.1 }
  \def \shiftMargin {0.3}
  \newcommand\numbers{{0.25, 0.6, 0.95, 1.3, 1.65, 2, 2.35, 2.7}}

  % Rectangle A
  \draw[rounded corners=2pt] (\xStartA, \yStartA) rectangle (\xStartA + \recWidthA, \yEndA)
  node [above, xshift=-.25cm] {\textarabic{◌َ}};
  % Populating the rectangle A
  \node at (0.25 + \xStartA, 3 -\numbers[0]) {1};
  \foreach \j in {1,...,3}
  \node at (0.25 + \xStartA, \yEndA -\numbers[\j]) {0};

  % Annotation A
  \node at (\xStartA-\shiftMargin, \yStartA + \half) {\small 4};
  \draw[arrows=-angle 90, very thin] (\xStartA-\shiftMargin, \yStartA + \half +.3) -- (\xStartA-\shiftMargin, \yEndA -0.1);
  \draw[arrows=angle 90-, very thin] (\xStartA-\shiftMargin, \yStartA + .1) -- (\xStartA-\shiftMargin, \yStartA + \half -.3);

  % Rectangle B
  \draw[rounded corners=2pt] (\xStartB +0.5, \yStartB) rectangle (\yStartB +0.5 -1.5, \yEndB)
  node[above, xshift=0.25cm] {\textarabic{ب}};
  \node at (\xStartB +0.25, \yEndB -\numbers[0]) {0};
  \node at (\xStartB +0.25, \yEndB -\numbers[1]) {1};
  \node at (\xStartB +0.25, \yEndB -\numbers[2]) {0};
  \node at (\xStartB +0.25, \yEndB -\numbers[3]) {0};
  \node at (\xStartB +0.25, \yEndB -\numbers[4] ) {\vdots};
  \node at (\xStartB +0.25, \yStartB +\numbers[0]) {0};

  % Annotation B
  \node at (-\shiftMargin, \yStartA + \half) {\small 37};
  \draw[arrows=-angle 90, very thin] (-\shiftMargin, \yStartA + \half +.3) -- (-\shiftMargin, \yEndB -0.1);
  \draw[arrows=angle 90-, very thin] (-\shiftMargin, \yStartB + .1) -- (-\shiftMargin, \yStartA + \half -.3);

  \node at (0.8, \yStartA +\half) {$+$};
  \node at (2.5 + .02, \yStartA +\half) {$=$};

  % Rectangle C
  \draw[rounded corners=2pt]  (\xStartC, \yStartC) rectangle (\xEndC, \yEndC)
  node[above, xshift=-0.25cm] {\textarabic{بَ}};
  ;
  \node at (\xStartC +0.25, \yEndC -\numbers[0]) {1};
  \node at (\xStartC +0.25, \yEndC -\numbers[1]) {0};
  \node at (\xStartC +0.25, \yEndC -\numbers[2]) {0};
  \node at (\xStartC +0.25, \yEndC -\numbers[3]) {0};

  \node at (\xStartC +0.25, \yEndC -\numbers[4] -.2) {0};
  \node at (\xStartC +0.25, \yEndC -\numbers[5] -.2) {1};
  \node at (\xStartC +0.25, \yEndC -\numbers[6] -.2) {0};
  \node at (\xStartC +0.25, \yEndC -\numbers[7] -.2) {\vdots};
  \node at (\xStartC +0.25, \yStartC +\numbers[0]) {0};

  % Annotation
  \node at (\xStartC -\shiftMargin, \yStartA +\half) {\small 41};
  \draw[arrows=-angle 90, very thin]  (\xStartC -\shiftMargin, 2.5) -- (\xStartC -\shiftMargin, \yEndC -0.1);
  \draw[arrows=angle 90-, very thin]  (\xStartC -\shiftMargin, 0.5) -- (\xStartC -\shiftMargin, 2);

  \draw [decorate,decoration={brace,amplitude=2pt,mirror,raise=2pt}, very thin]
  (\xEndC +\shiftMargin -.1, 2.5 +.2) -- (\xEndC +\shiftMargin  -.1, \yEndC -0.1);
%  node [right, midway, xshift=0.2cm, yshift=0.1cm] {\tiny $4 \times 1$ diacritic vector};

  \draw [decorate,decoration={brace,amplitude=2pt,mirror,raise=2pt}, very thin]
  (\xEndC +\shiftMargin -.1, 0.5) -- (\xEndC +\shiftMargin -.1, 2.4);
%  node [right, midway, xshift=0.2cm, yshift=0.1cm] {\tiny $37 \times 1$ letter vector};


  \node [rectangle, text width= 5em,font=\footnotesize] at (5.6, 3.4)
    {\scriptsize $4 \times 1$ letter vector which represents \textarabic{◌َ}};

  \node [rectangle, text width= 5em,font=\footnotesize] at (5.6, 1.4)
    {\scriptsize $37 \times 1$ letter vector which represents \textarabic{ب}};

  % \node at (0.25, 0.5) {$\bm m$};
  % \node at (1.5 +.25, 0.5 +0.5) {$\bm k$};
\end{tikzpicture}

%%% Local Variables:
%%% mode: latex
%%% TeX-master: "../main"
%%% End:

};
    \node at (-12+0.3,-3){(a) \textit{One-Hot}};
    \node at (-7+0.3, -3){(b) \textit{Binary}};
    \node at (0-0.5,  -3){(c) \textit{Two-Hot}};
\end{tikzpicture}


\caption{Different encoding mechanisms%: \textit{One-hot}, \textit{binary}, and \textit{two-hot}  encoding. Using the word example %\textarabic{مَرْحَبَا} \textit{One-hot} encoding is applied using $n$-letter alphabet $n = 181$ in Arabic. Same word encoded using \textit{binary} encoding where the vector length $n = \ceil*{\log_2 l}$, $l \in \{181\}$. \textit{two-hot} encoding is applied by stacking $\bm k_{4 \times 1}$ on the top of $\bm m_{37 \times 1}$ we get the $Two-Hot_{41 \times 1}$ \usebox{\columnVector}, which represents a letter and its diacritic simultaneously. 
}~\label{Fig:One_Binary_Encoding}
\end{figure*}



\begin{description}

\item[\textbf{One-Hot encoding}] In this encoding system, We assume the letter with the diacritic as one unit. So, for example, \textarabic{د} represented as letter differs than (\textarabic{دَ, دِ, دُ, دْْ}). Now every letter is represented 5 times one without diacritic and four times with different diacritics combinations $36 \times 5$ besides the white-space character. So, the total is $181 \times 1$. From now forward, We have 181-characters Arabic alphabet represent the One-Hot encoding, and according to it, we will encode verses. (Figure~\ref{Fig:One_Binary_Encoding}).

We need to mention that, One-Hot encoding technique is of the famous techniques in encoding problem. We will not compare the encoding technique in these sections. However, We will discuss it in details in model results. Also, The implementations of the One-Hot is trivial. But we need to focus on here about the size for every letter which is $181 \times 1$ which means if we have a verse with 82 characters it will results up with a matrix $181 \times 82$ which is very big to be in memory.

\item[\textbf{Binary Encoding}] The idea is to represent a letter with an $n \times 1$ vector which contains a unique combination of ones and zeros.  $n =\ceil*{\log_2} l$ where $l$ is the alphabet length, and $n$ is the sufficient number of digits to represent $l$ unique binary combinations.  For example a phrase like this \textarabic{مَرْحَبا}, it has 5 characters, figure~\ref{Fig:One_Binary_Encoding} shows how it is encoded as a $8 \times 5$ matrix, which saves 22.6 times memory than the \textit{one-hot} and reduces the model input dimensions, significantly. But on the other hand, the input letter share some features between them due to the binary representation as it is shown in figure~\ref{Fig:One_Binary_Encoding}.

\item[\textbf{Two-Hot encoding}] This is an intermediate technique which takes the advantages of the previous two encoding techniques. In which we encode the letter and its diacritic separately as two \textit{One-Hot} vectors, this way the letter is encoded as $37 \times 1$ \textit{One-Hot} vector and the diacritic is encoded as $4 \times 1$ \textit{One-Hot} vector, then both vectors stacked to form one $41 \times 1$ vector (Figure~\ref{Fig:One_Binary_Encoding}).

By this way, we reduced the vector dimension from 181 to 41 and also minimizes the number of shared features between vectors to the maximum one at each vector. 
\end{description}




%%% Local Variables:
%%% mode: latex
%%% TeX-master: "../master"
%%% TeX-engine: xetex
%%% End:
