\chapter{\uppercase{Introduction}}\label{Ch:Intro}
In this chapter, we give an introduction and basic background knowledge on Arabic language. Some details on Arabic Poetry history will be provided, then the aim of this thesis will be stated.

\section{Thesis Outline}
This thesis is arranged as follows:
\begin{enumerate}
	\item Chapter 1 presents some basic introduction and background knowledge on Arabic Poem and its definitions. Additionally, it contains details about the Arabic language and some features used during our work.
	\item Chapter 2 presents a background related to Al-Arud, Deep Learning fundamentals,
	\item Chapter 3 presents a literature review for the previous work on this topic.
	\item Chapter 4 details the dataset acquisition. It introduces the
	dataset acquisition and encoding, including the essential pre-processing steps, and the
	justification for their need. Pre-processing steps are data extraction, data cleansing, data
	format, and different techniques for data encoding. The chapter contains comparisons among the
	three encoding techniques. 
	\item Chapter 5 presents model training and experiment design. The chapter also presents the details of the model selection, the architecture and hyper-parameters.
	\item Chapter 6 presents the results and discussion. It contains the details of the 192 experiments done in this project. Additionally, the chapter presents the details of results’ analysis and discussion.
	\item Chapter 7 presents the conclusion and proposed future work.
\end{enumerate}



\section{Arabic Poetry } %%
Arabic is the fifth most widely spoken language~\cite{Ethnologue_2017}. It is written from right to left. Its alphabet consists of 28 primary letters, 8 letters derived from the basic ones, so the total count of Arabic characters is 36 characters. The writing system is cursive; hence, most letters join to the letter that comes after them, although a few letters remain unconnected.

Arabic poetry(\textarabic{الشعر العربى}) is the earliest form of Arabic literature. It dates back to the sixth century. Poets have written poems without knowing exactly what rules render a collection of words a poem. People recognize poetry by nature, but only talented ones can write poems. This was the case until \textit{Al-Farahidi} (718 – 786 CE) analyzed the Arabic poetry; he noted that the succession of consonants and vowels produce patterns or \textit{meters}, which make the music of poetry. He counted fifteen meters; thereafter, a student of him added one more meter to make sixteen. Arabs call meters \textarabic{بحور} which means ``\textit{seas}''. The study of Arabic Poetry Meter Classification is named \textit{Al-Arud} (\textarabic{العَروض}). It takes much time to become an expert in this field. 
\section{Deep Learning}

Deep Learning, also named Deep Neural Network, is part of Machine Learning algorithms. Deep Learning attempts to simulate the human brain into neural dependency.  By using Deep Learning, we can achieve better results when learning from the data. Deep Neural Network needs a huge amount of data to achieve the expected learning curve and results. It also needs a massive amount of computation to train. We used the Recurrent Neural Network (RNN) to work on the Arabic text and show its ability to achieve outstanding performance on problems of text data. We also used LSTM to solve the long dependency issue in RNN~\nameref{Sec:Deep_Learning_Background}.

\section{Thesis Objectives}

This thesis aims to explore the Arabic language feature using established Al-Arud rules. This exploration helps us understand how to use a machine learning model to understand Arabic language features without hand-crafted features or static rule-based models.

Artificial Intelligence currently works to do the human tasks such as our problem here. Our target in this thesis is to achieve the human accuracy which will make it easy for anyone to recognise the meter for any poem without referring to the language experts or to study the whole field.

In this thesis, we explain how to use deep learning to classify the Arabic poem. We also explain in detail the features of Arabic poem and how to deal with these. We explain how anyone can work with Arabic text encoding in a dynamic way to encode the text at the character level and deal with Arabic text features such as the Tashkeel. We were able to achieve classification accuracy score of 0.9638 for classifications of the sixteen Arabic meter classes.

This is a basic step to get deeper into the language and, later not only can we have more robust machine learning models which understand and classify the Poem/poetry, but we can also generate a similar text with the model without breaking the rules. All the previous are first steps toward a deeper understanding of the Arabic language features and rules using deep learning techniques. This research can open ideas about how machine understanding of Arabic Grammar rules leads to new ideas to generate better Arabic text models.

This research is the first to address classifying poem meters via a machine learning approach. It aims to have a deep understanding of Arabic language grammar and features in the Poem meters classification and is a base for further research related to the poem analysis problem.

In this study, we work on Poetry Meter Classification and use the latest technologies to check the class of poem. We also work to achieve near human expert results which make our work a breakthrough in the field concerning the results compared to previous results. Figure~\ref{Fig:Thesis_Cycle} shows the steps. First, we derived the data from the available sources with labeling. Second, the clean data was transformed and encoded. Finally, the cleaned data was trained, validated, and enhanced into the RNN model.
\begin{figure}[!ht]
	   \begin{tikzpicture}[>=latex']
        \tikzset{block/.style= {draw, rectangle, align=center,minimum width=2cm,minimum height=1cm},
        rblock/.style={draw, shape=rectangle,rounded corners=1.5em,align=center,minimum width=2cm,minimum height=1cm},
        input/.style={ % requires library shapes.geometric
        draw,
        trapezium,
        trapezium left angle=60,
        trapezium right angle=120,
        minimum width=2cm,
        align=center,
        minimum height=1cm
    },
        }
        \node [rblock]  (start) {Start};
        \node [block, right =1cm of start] (crawl) {Data Crawling};
        \node [block, right =1cm of crawl] (clean) {Data Cleansing};
        \node [block, right =1cm of clean] (encode) {Data Encoding};
        \node [block, below right =2cm and -0.5cm of start] (train) {Training};
        \node [block, right =1cm of train] (validate) {Validation \& Testing};
        \node [block, right =1cm of validate] (choose) {select Best Model};
        \node [rblock, right =1cm of choose] (end) {End};
        \node [coordinate, below right =1cm and 1cm of encode] (right) {};  %% Coordinate on right and middle
        \node [coordinate,above left =1cm and 1cm of train] (left) {};  %% Coordinate on left and middle
        \node [coordinate,below  =1cm and 1cm of validate] (loop) {};  %% Coordinate on left and middle

%% paths
        \path[draw,->] (start) edge (crawl)
                    (crawl) edge (clean)
                    (clean) edge (encode)
                    (encode.east) -| (right) -- (left) |- (train)
                    (train) edge (validate)
                    (validate) -- (loop) -| (train)
                    (validate) edge (choose)
                    (choose) edge (end)
                    ;
    \end{tikzpicture}

%%% Local Variables:
%%% mode: latex
%%% TeX-master: t
%%% End:

	\caption{Thesis Working Steps.}
	\label{Fig:Thesis_Cycle}
\end{figure}




%%% Local Variables:
%%% mode: latex
%%% TeX-master: "../master"
%%% TeX-engine: xetex
%%% End:
