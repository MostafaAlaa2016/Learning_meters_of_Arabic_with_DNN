\chapter{\uppercase{Introduction}}\label{Ch:Intro}
In this chapter, we give an introduction and basic background knowledge on Arabic language. Some details on Arabic Poetry history will be provided, then the aim of this thesis will be stated.

\section{Arabic Poetry } %%
Arabic is the fifth most widely spoken language~\cite{Ethnologue_2017}. It is written from right to left. Its alphabet consists of 28 primary letters, 8 letters derived from the basic ones, so the total count of Arabic characters is 36 characters. The writing system is cursive; hence, most letters join to the letter that comes after them, although a few letters remain unconnected.

Arabic poetry(\textarabic{الشعر العربى}) is the earliest form of Arabic literature. It dates back to the sixth century. Poets have written poems without knowing exactly what rules render a collection of words a poem. People recognize poetry by nature, but only talented ones can write poems. This was the case until \textit{Al-Farahidi} (718 – 786 CE) analyzed the Arabic poetry; he noted that the succession of consonants and vowels produce patterns or \textit{meters}, which make the music of poetry. He counted fifteen meters; thereafter, a student of him added one more meter to make sixteen. Arabs call meters \textarabic{بحور} which means ``\textit{seas}''. The study of Arabic Poetry Meter Classification is named \textit{Al-Arud} (\textarabic{العَروض}). It takes much time to become an expert in this field. 
\section{Deep Learning}

Deep Learning, also named Deep Neural Network, is part of Machine Learning algorithms. Deep Learning attempts to simulate the human brain into neural dependency.  By using Deep Learning, we can achieve better results when learning from the data. Deep Neural Network needs a huge amount of data to achieve the expected learning curve and results. It also needs a massive amount of computation to train. We used the Recurrent Neural Network (RNN) to work on the Arabic text and show its ability to achieve outstanding performance on problems of text data. We also used LSTM to solve the long dependency issue in RNN~\ref{Sec:Deep_Learning_Background}.

\section{Thesis Objectives}
This thesis aims to explore the Arabic language feature using established Al-Arud rules. This exploration helps us understand how to use a machine learning model to understand Arabic language features without hand-crafted features or static rule-based models.

This is a basic step to get deeper into the language and, later not only can we have more robust machine learning models which understand and classify the Poem/poetry, but we can also generate a similar text with the model without breaking the rules. All the previous are first steps toward a deeper understanding of the Arabic language features and rules using deep learning techniques. This research can open ideas about how machine understanding of Arabic Grammar rules leads to new ideas to generate better Arabic text models.

This research is the first to address classifying poem meters via a machine learning approach. It aims to have a deep understanding of Arabic language grammar and features in the Poem meters classification and is a base for further research related to the poem analysis problem.

In this study, we work on Poetry Meter Classification and use the latest technologies to check the class of poem. We also work to achieve near human expert results which make our work a breakthrough in the field concerning the results compared to previous results. Figure~\ref{Fig:Thesis_Cycle} shows the steps. First, we derived the data from the available sources with labeling. Second, the clean data was transformed and encoded. Finally, the cleaned data was trained, validated, and enhanced into the RNN model.

\begin{figure}[!t]
   \input{./Figures/Ch_1_Intro/Fig_Master_Cycle.tex}
  \caption{Thesis Working Steps.}
  \label{Fig:Thesis_Cycle}
\end{figure}

% @@@
% ``will''; e.g., ``In this chapter, we will give ...'' => `` we give...''

% be consistent in formatting: you wrote ``meters'' using italic then wrote ``Al-arud'' using bold face without italic. No need for bold face; just italic is enough because it is a new term.

% The following statements need complete re-writing; I left them unchanged: ``This thesis aims to
% explore the Arabic language feature using established Al-Arud rules. This exploration will help us
% understand how to use a machine learning model to understand Arabic language features without
% hand-crafted features or static rule-based models.''

% I left the rest of the section ``Thesis Objective'' without revision, since I fed up.

% General comments to this guy: Many statements need re-phrasing or re-wording. Some statements need
% to be merged. Some other places include redundant phrasing.


%%% Local Variables:
%%% mode: latex
%%% TeX-master: "../master"
%%% TeX-engine: xetex
%%% End:
