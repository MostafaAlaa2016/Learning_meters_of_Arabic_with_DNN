\chapter{Conclusion and Future Work}\label{cha:concl-future-work}



\section{Conclusion}\label{sec:conclusion}

This thesis aimed to train Recurrent Neural Networks on Arabic Poems at character level to recognize their meters. This can be considered a step forward for language understanding, synthesis, and style recognition. This step aimed to understand how Deep Learning models can understand the Arabic Poem Meters and its rules and grammar. The datasets were crawled from several non-technical online sources; then cleaned, structured, and published to a repository that is made publicly available for scientific research. To the best of our knowledge, using Machine Learning (ML) and Deep Neural Networks (DNN), in particular for learning poem meters and phonetic style from a written text, along with the availability of such a dataset, is new to the literature. We described different encoding mechanisms, each with its own structure and understanding mechanism. We also experimented different RNN configurations including number of layers, cell units and types, with different language settings (with and without \textit{diacritics}) We endeavoured to use different ways to handle the imbalanced dataset, showing the different accuracy and the effect of each one. The classification accuracy obtained on our Arabic Poem dataset was remarkable, especially if compared to that obtained from the deterministic and human-derived rule-based algorithms available in the literature.

\clearpage

\section{Future Work}

In this section, we outline future work which can be undertaken based on this research. We will divide the proposed future work into two parts; the first related to this idea and how can we enhance it, and the second, related to the new research area which can be built on our dataset.

\begin{itemize}
  \item Enhancement on the current work:
  \begin{enumerate}
    \item The classification results could be enhanced to be the same level as a human expert. We have many areas of enhancement; first, enhancing the network configurations with more layers with combinations of cell units and batch size. Secondly, there is an open area to solve the accuracy drops in the per-class performance issue in the small classes as following: 
    {\color{red}
    	\begin{itemize}
    		\item New design of weighting loss function. 
    		\item Increase the dataset for the small classes, which will affect the learning and understanding of their patterns. 
    		\item Train the dataset with equal amount of classes which named desampling. 
    		\item Dublicate some poem in the small classes.
    	\end{itemize}
}
    \item This problem can be treated as unsupervised learning, which will be a different approach to the problem-solving.
  \end{enumerate}
  \item Build new work based on the current dataset:
    \begin{enumerate}
    \item Classify the correct and incorrect poems, respecting the nearest meter.
    \item Identify the issues in the written poems regarding the meter rules, and attempt to address them.
    \item Classify the poetry’s meaning, as this paper did not work for this idea.
    \item Generate a new poem from learning the current classes and patterns.
    \item Analyze the historical impact on the poems and the poetry, for example for a specific period, if the poetry is affected by this period, or if there are patterns of writing between the poetry.
  \end{enumerate}
  
\end{itemize}


%%% Local Variables:
%%% mode: latex
%%% TeX-master: "../master"
%%% TeX-engine: xetex
%%% End:
