\chapter{Conclusion and Future Work}\label{cha:concl-future-work}



\section{Conclusion}\label{sec:conclusion}

This thesis aimed to train Recurrent Neural Networks on Arabic Poem at character level to recognize their meters. This can be considered a step forward for language understanding, synthesis, and style recognition. This step aimed to understand how can Deep Learning models understand the Arabic Poem Meters and its rules and grammar. The datasets were crawled from several non-technical online sources; then cleaned, structured, and published to a repository that is made publicly available for scientific research. To the best of our knowledge, using Machine Learning (ML) and Deep Neural Networks (DNN) in particular for learning poem meters and phonetic style from a written text, along with the availability of such a dataset, is new to literature. We described different encoding mechanism each has its own structure and understanding mechanism. We Also experiment different RNN configurations including (number of layers, cell units, cell types,..) with different language settings (with Diacritics and without Diacritics) We tried to use different ways to handle the imbalanced dataset which shows the different accuracy and the effect of each one. The classification accuracy obtained on our Arabic Poem dataset was remarkable, especially if compared to that obtained from the deterministic and human-derived rule-based algorithms available in the literature.

\clearpage

\section{Future Work}

In this section, We will mention some future work which can be built based on this research. We will split the future work into two parts, One related to this idea and how can we enhance it. Second, related to the new research area which can be built on the dataset we have.

\begin{itemize}
  \item Enhancement on the current work
  \begin{enumerate}
    \item Enhance the classification results to be the same as the human expert. We have many areas of enhancement. First, Enhance the network configurations with more layers with combinations of cell units and batch size. Second, There is an open area to solve the accuracy drops in the Per-class performance issue in the small classes using a new design of weighting loss function. Third, It can increase the dataset for the small classes which will affect the learning and understanding of their patterns.
    \item This problem can be treated as unsupervised learning which will be a different approach to the problem-solving.
  \end{enumerate}
  \item Build new work based on the current dataset
    \begin{enumerate}
    \item {\color{red} Classify the correct and non-correct poem respecting the nearest meter.
    \item Identify the issue in the written poem and try to solve it regarding the meter rules.}
    \item Classify the poetry meaning as this paper did not work for this idea.
    \item Generate a new poem from learning the current classes and patterns.
    \item Analyze the historical impact on the Poem and the Poetry for example for a specific period if the poetry affected by this period, or there are patterns of writing between the Poetry or not.
  \end{enumerate}
  
\end{itemize}


%%% Local Variables:
%%% mode: latex
%%% TeX-master: "../master"
%%% TeX-engine: xetex
%%% End:
