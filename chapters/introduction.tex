\chapter{\uppercase{Introduction}}


%\begin{spacing}{1.5}\section*{\LARGE{\mdseries{Thesis Outline}}}

 

  Arabic is the fifth most widely spoken language\footnote{\textit{according to the 20th edition of Ethnologue, 2017}}. It is written from right to left. Its
alphabet consists of 28 primary letters, and there are 8 more derived letters
from the basic ones, so the total count of Arabic characters is 36 characters.
The writing system is cursive; hence, most letters join to the letter that comes
after them, a few letters remain disjoint.

Each Arabic letter represents a consonant, which means that short vowels are not
represented by the 36 characters, for this reason, the need of \textit{diacritics}
rises. \textit{Diacritics} are symbols that comes after a letter to state the
short vowel accompanied by that letter. There are four diacritics \textarabic{◌َ} \textarabic{◌ُ}
\textarabic{◌ِ} \textarabic{◌ْ} which represent the following short vowels
/\textit{a}/, /\textit{u}/, /\textit{i}/ and \textit{no-vowel} respectively,
their names are \textit{fat-ha, dam-ma, kas-ra and sukun} respectively.  The first
three symbols are called \textit{harakat}. Table \ref{tables:diacritics_dal}
shows the 4 diacritics on a letter.



% table: dal with diacritics
\begin{table}[H]
	\centering
	\begin{tabular}{c c c c c c}
		%\hline
		\toprule
		\textbf{\small{Diacritics}}     & \small{\textit{without}} & \small{\textit{fat-ha}} &
		\small{\textit{kas-ra}} & \small{\textit{dam-ma}} & \small{\textit{sukun}}\\
		%\hline
		\midrule
		\textbf{\small{Shape}}   & \textarabic{د} & \textarabic{دَ} & \textarabic{دِ} &
		\textarabic{دُ} & \textarabic{دْ}\\
		%\hline
		\bottomrule
	\end{tabular}
	\caption{\textit{Diacritics on the letter  \textarabic{ د }}}\label{tables:diacritics_dal}
\end{table}



There are two more sub-diacritics made up of the basic four to represent two
cases:
\begin{definition}\label{def:shadaa_definition}
  \textbf{Shadaa}  \hfill \\
to indicate the letter is doubled. Any letter with
shaddah (\textarabic{ ّ } ) the letter should be duplicated: first letter with a
constant (sukoon) and second letter with a vowel (haraka) \cite{Alnagdawi2013}; Table  \ref{tables:shadda_dal}
shows the dal with shadda and the original letters.
% table: dal with shadda

\begin{table}[H]
	\centering
	\begin{tabular}{c c c}
		%\hline
		\toprule
		\textbf{\small{Diacritics}} & \small{\textit{letter with Shadda }} & \small{\textit{letters without shadaa  }} \\
		%\hline
		\midrule
		\textbf{\small{Shape}}  & \textarabic{دَّ} &  \textarabic{دْدَ}\\
		%\hline
		\bottomrule
	\end{tabular}
	\caption{\textit{Shadaa diacritics on the letter  \textarabic{ د }}}\label{tables:shadda_dal}
\end{table}

\end{definition}

\begin{definition}\label{def:tanween_definition}
  \textbf{Tanween} \hfill \\
  %%% \ref{defa} and \ref{defb}
  is doubling the short vowel, and can convert
Tanween fathah, Tanween dhammah or Tanween kasrah by
replacing it with the appropriate vowel ( ُ◌ – dhammah, َ◌ –
fathah or ِ◌ –kasrah ) then add the Noon letter with constant to the end of the word \cite{Alnagdawi2013}. Table \ref{tables:Tanween_dal}
shows the difference between the original letter and the letter with Tanween

\begin{table}[H]
	\centering
	\begin{tabular}{c c c}
		%\hline
		\toprule
		\textbf{\small{Diacritics}} & \small{\textit{letter with tanween }} & \small{\textit{letters without tanween}} \\
		%\hline
		\midrule
          
          \textbf{\small{Tanween Fat-ha}}  & \textarabic{دً} &  \textarabic{دَ+نْ}\\
          \textbf{\small{Tanween Dam-ma}}  & \textarabic{دٌ} &  \textarabic{دُ+نْ}\\
          \textbf{\small{Tanween Kas-ra}}  & \textarabic{دٍ} &  \textarabic{دِ+نْ}\\
          
	
		\bottomrule
	\end{tabular}
	\caption{\textit{Tanween diacritics on the letter  \textarabic{ د }}} \label{tables:Tanween_dal}
\end{table}


\end{definition}

 Arabs pronounce the sound \textit{/n/} accompanied \textit{sukun} at the end the indefinite words, that sound corresponds to this
letter \textarabic{نْ}, it is called \textit{noon-sakinah}, however, it is
just a phone, it is not a part of the indefinite word, if a word comes as a
definite word, no additional sound is added. Since it is not an essential sound,
it is not written as a letter, but it is written as  \textit{tanween}
\textarabic{◌ٌ ◌ً ◌ٍ}.
% adding tanween and its relationship to the previous letter
\textit{Tanween} states the sound \textit{noon-sakinah}, but as you have noticed,
there are 3 \textit{tanween} symbols, this because  \textit{tanween} is added as
a diacritic over the last letter of the indefinite word, one of the 3 harakat\textit{harakat} accompanies the last letter, the last letter's \textit{harakah}
needs to be stated in addition to the sound \textit{noon-sakinah}, so
\textit{tanween} is doubling the last letter's \textit{haraka}, this way the last
letter's \textit{haraka} is preserved in addition to stating the sound
\textit{noon-sakinah}; for example, \textarabic{رَجُلُ + نْ} is written
\textarabic{رَجُلٌ} and  \textarabic{رَجُلِ + نْ} is written \textarabic{رَجُلٍ}. 


Those two definition, Definition ~\ref{def:shadaa_definition} and Definition ~\ref{def:tanween_definition}  will help us to reduce the dimension of the letter's feature vector as we will see in \textit{preparing data} section.


Diacritics makes short vowels clearer, but they are not necessary.
Moreover, a phrase without full diacritics or with just some on some letters is
right linguistically, so it is allowed to drop them from the text.

% Diacritics in Unicode
In Unicode, Arabic diacritics are standalone symbols, each of them has its own
unicode. This is in contrast to the Latin diacritics; e.g., in the set
\textit{\{ê, é, è, ë, ē, ĕ, ě\}}, each combination of the letter \textit{e} and a diacritic is represented by one unicode.

\newpage

\section{Arabic Arud Science}
reference is the book .... to be added
\begin{definition}\label{def:arud}
  \textbf{Arud} \hfill \\
In Arabic Arud natively has many meanings (the way, the direction, the light clouds and Mecca and Madinah\footnote{\textit{Mecca and Madinah are two cities in  Saudi Arabia}.}. Arud is the science which studies The Arabic Poem meters and the rules which confirm if the Poem is sound meters \& broken meters. It named Arud because some people said he put this science in Arud place \textarabic{العَروض} \textit{with fat-ha, not with dam-ma such as the science name \textarabic{العُروض} } between Mecca and Madinah.
\end{definition}

The Author of this science is \textit{Al-Farahidi} (718 – 786 CE) has analyzed the
Arabic poetry; then he came up with that the succession of consonants and vowels
produce patterns or \textit{meters}, which make the music of poetry. He was one of the famous people who know The melodies and the musical parts of speech. He has
counted them fifteen meters.  After that, a student of \textit{Al-Farahidi} has
added one more meter to make them sixteen. Arabs call meters \textarabic{بحور}
which means "\textit{seas}." Poets have written poems without knowing exactly what rules which make a collection of words a poem.

The Reasons which makes \textit{Al-Farahidi} put this science is

  \begin{itemize}
  \item Protect the Arabic Poems from the broken meters.
  \item Distinguish between the original Arabic Poem and the non-poem or from the prose.
    \item Make the rules clear and easy for anyone who needs to write a poem.
  \end{itemize}
  
    Some people said that the one-day Al-Farahidi was walking into the metal-market and he was said some of the poems and for some reasons the knock of the metals matched the musical sound of the poem he was saying then he got an idea to explore the Arud of the poems.
\newpage
    \subsection{Feet Representation}
    A meter is an ordered sequence of feet. Feet are the basic
units of meters; there are ten of them.
\begin{definition}\label{def:feet}
  \textbf{Feet} \hfill \\  A Foot consists of
a sequence of \textbf{Sukun} (Consonants) represented as (0) and \textbf{Harakah} (Vowels) (/). Traditionally, feet are represented by mnemonic words called tafa’il \textarabic{تفاعيل}.
\end{definition}

Feets consists of three parts (Reasons \textarabic{أسباب}, Wedge \textarabic{وتد}, Breaks \textarabic{فواصل}).
\begin{itemize}
\item \textbf{Reasons (\textarabic{أسباب})}: It has two types
  \begin{enumerate}
  \item \textbf{Light (\textarabic{سبب خفيف})} which happens when we have the first letter is harakah and the second is sukun (/0) example (\textarabic{هَبْ, لَمْ}).
    \item \textbf{Heavy (\textarabic{سبب ثقيل})} which happens when we have two harakah letter (//) example (\textarabic{لَكَ, بِكَ}).
    \end{enumerate}
    \item \textbf{Wedge (\textarabic{وتد})}: It has two types
  \begin{enumerate}
  \item \textbf{Combined Wedge (\textarabic{وتد مجموع})} which happens when we have two harakah letters followed by sukun (//0) example (\textarabic{مَشَى, عَلَى}).
    \item \textbf{Separated Wedge (\textarabic{وتد مفروق})} which happens when we have two harakah and in between a sukun letter (/0/) example (\textarabic{مُنْذُ, مِصْرُ}).
    \end{enumerate}
    \item \textbf{Breaks (\textarabic{فواصل}}): It has two types
  \begin{enumerate}
  \item \textbf{Small Break (\textarabic{فاصلة صغرى}}) which happens when we have three harakah letters followed by a sukun letter (///0) example (\textarabic{ذَهَبُوا, سُفُناً}).
    \item \textbf{Big Break (\textarabic{فاصلة كبرى}}) which happens when we have four harakah letters followed by a sukun letter  (////0) example (\textarabic{جَعَلَهُمْ}).\footnote{\textit{Some of Arab linguistic scientist assume the small Breaks as a combination between big reason and small reason. Same for the Big Breaks assumed to be a combination between Big reason and Combined Wedge. So, they didn't assume we have three types of feet it is only pure two and any other feets constructed from this two. In this thesis we assume there are three feets }.}
    \end{enumerate}
  \end{itemize}
  
\newpage
  \subsubsection{Rules for Arabic Letters Representation}
  Arabic letter has one general rule in the poem representation which is we represent only the letters which is (spoken) not the written which means the letters with phonatics not the written. We have give the below rules as a results of the general rule.

  \begin{itemize}
  \item Any letter with \textit{harakah} represented as (/).
  \item Any letter with \textit{sukun} represented as (0).
  \item Any letter with shaddah represented by two letters the first one will be \textit{sukun} and the second letter will be \textit{harakah} represented as (0/) example (\textarabic{مُحَمََّد}) will be (//0//0).
  \item Any letter with tanween represented by two letters the first one is \textit{haraka} (/) and the second is \textit{sukun}.
  \item Alef without hamze (\textarabic{همزة الوصل}) and Wow Algmaa are not represented example (\textarabic{وُاعلَموا}) will be (/0//0)
  \item If we have a letter which is not written but (spoken) so, we will represent it example (\textarabic{هذا}) it include Alef but not written (\textarabic{هاذا}) the representation will be (/0/0).
  \item If we have \textit{Meem Aljamaa} with harakah so, it represented with \textit{Mad} example (\textarabic{هُمُ}) will be (//0) .
  \item \textit{Alef Mad} (\textarabic{آ}) will be two letters \textit{Alef with harakah} and \textit{Alef with sukun} example (\textarabic{آدَمُ}) will be (/0//).
    \item if the verse ended with \textit{harkah} we will add \textit{sukun} to it.
    
    
    \end{itemize}
Exampel: (note: the below representation is not complete )
\begin{Arabic}
	\begin{traditionalpoem*}
وَجَدتُ الناسَ مَيتاً مِثلَ حَيٍّ *** بِحُسنِ الذِكرِ أَو حَيّاً كَمَيتِ
وَجَدْتُنْنَاْ سَمَيْتَنْمِثْ لَحَيْيِنْ *** بِحُسْنِذْذِكْ رِأَوْحَيْيَنْ كَمَيْتِيْ

	\end{traditionalpoem*}
\end{Arabic}
    
\newpage

\subsection{Arabic Poetry Feets}

Arabic poetry feets has ten tafa'il \textarabic{تفاعيل} (scansion)  any peom constructed from these feets. They are eight from writing (syntax) prespective, But it ten in the rules.
\begin{savenotes}

\begin{table}[H]
  \centering
  \begin{tabular}{|c|c|c|c|}
    \hline
    \textbf{\#} & \textbf{Feet} & \textbf{Scansion} & \textbf{Construction} \\
    \hline
    1 & \textarabic{فَعُولُنْ}  & \texttt{0/0//} & combined wedge (\textarabic{فعو}) and small reason (\textarabic{لن})   \\
    2 &\textarabic{مَفاعِيلُنْ}& \texttt{0/0/0//} & combined wedge (\textarabic{مفا}) and two light reasons (\textarabic{عي}) (\textarabic{لن})   \\
    3 &\textarabic{مُفَاعَلَتُنْ}& \texttt{0///0//}  &    combined wedge (\textarabic{مفا}), heavy reason (\textarabic{عل}) and light reason (\textarabic{تن}) \\
    4 &\textarabic{فَاعِلاَتُنْ} & \texttt{0/0//0/}   & light reason (\textarabic{فا}), combined wedge (\textarabic{علا}) and light reason (\textarabic{تن})   \\
    5 &\textarabic{فَاعِ لاتُنْ} & \texttt{0/0//0/}  &  Separated wedge (\textarabic{فاع}) and two light reason (\textarabic{لا})(\textarabic{تن}) \footnote{\textit{We separated the letters (\textarabic{ع}) and (\textarabic{لا}) in (\textarabic{فاع لاتن}) to show that this part is separated wedge and distinguish between this feet  and (\textarabic{فاع لاتن}) which contains combined wedge  }.}  \\    
    6 &\textarabic{فَاعِلُنْ}  & \texttt{0//0/}   & light reason (\textarabic{فا}) and combined wedge (\textarabic{علن})\\
    7 &\textarabic{مُتَفَاعِلُنْ}& \texttt{0//0///}  & heavy reason (\textarabic{مت}), light reason (\textarabic{فا}) and combined wedge (\textarabic{علن})  \\
    8 &\textarabic{مَفْعُولاَت} & \texttt{0//0///}   & two light reason (\textarabic{مف})(\textarabic{عو}) and separated wedge (\textarabic{لات}) \\
    9 &\textarabic{مُسْتَفْعِلُنْ} & \texttt{0//0/0/}  &  two light reason (\textarabic{مس})(\textarabic{تف}) and combination wedge (\textarabic{علن}) \\
    10 &\textarabic{مُسْتَفْعِ لُنْ} & \texttt{0//0/0/}  & light reason (\textarabic{مس}), separated wedge  (\textarabic{تفع}) and light reason  (\textarabic{لن})\footnote{\textit{We separated the letters (\textarabic{ع}) and (\textarabic{ل}) in (\textarabic{مستفع لن}) to show that it ends with a separated wedge and distinguish between this feet  and (\textarabic{مستفعلن}) which contains combined wedge }}\\
    
    
    \hline
  \end{tabular}
  \caption{The ten feet of the Arabic meters. }\label{arud:feet}
\end{table}
    \end{savenotes}

%% Every digit (\texttt{/} or \texttt{0}) represents
%%    the corresponding diacritic over a letter in the feet. \texttt{/} corresponds to
%%    a\textit{harakah} ( \textarabic{◌َ}, \textarabic{◌ُ}, or \textarabic{◌ِ}) and \texttt{0}
%%    corresponds to a \textit{sukun} (\textarabic{◌ْ}). Any \textit{mad} (\textarabic{و, ا, ى}) is
%%    equivalent to \texttt{0}, \textit{tanween} is equivalent to \texttt{0/}, and \textit{shaddah} is
%%    equivalent to \texttt{/0}%

\newpage


\section{Arabic Poetry } %%
%% Introduction; the circumstances before alarud
Arabic poetry \textarabic{الشعر العربى} is the earliest form of Arabic literature. It dates back to the sixth century. Poets have written poems without knowing exactly what rules which make a collection of words a poem. People recognize poetry by nature, but only talented ones can write poems. This was the case until \textit{Al-Farahidi} (718 – 786 CE) has analyzed the
Arabic poetry, then he came up with that the succession of consonants and vowels
produce patterns or \textit{meters}, which make the music of poetry.  He has
counted them fifteen meters.  After that, a student of \textit{Al-Farahidi} has
added one more meter to make them sixteen. Arabs call meters \textarabic{بحور}
which means "\textit{seas}".

\begin{definition}\label{def:meter}
  \textbf{Meter} \hfill \\
  %%%% What is rtythm,feet
  Poetic meters define the basic rhythm of the poem. Each meter is described by a set of ordered feet which can
be represented as ordered sets of consonants and vowels \cite{Almuhareb2015}.

% \textbf{Some conventions and terminologies}:

% What are poems and terminologies?
% What does a poem look like? bayt, shatr, ....
\begin{Arabic}
	\begin{traditionalpoem*}
          ولد الهدى فالكائنات ضياء *** وفم الزمان تبسم وثناء انشاء
          الروح والملأ الملائك حوله *** للدين والدنيا به بشراء

	\end{traditionalpoem*}
\end{Arabic}%



\end{definition}


\begin{definition}\label{def:verse}
  \textbf{Arabic Verse} \hfill \\ refers to "poetry" as contrasted to prose. Where the common unit of a verse is based on meter or rhyme, the common unit of prose is purely grammatical, such as a sentence or paragraph \footnote{\textit{ https://en.wikipedia.org/wiki/Verse\_(poetry)}.}. A verse know as \textit{Bayt} in Arabic \textarabic{بيت}


\end{definition}


\begin{definition}\label{def:shatr}
  \textbf{Shatr} \hfill \\  A verse consists of two halves, each of them is called \textit{shatr} and carries the full meter.  We will use the term \textit{shatr} to refer to a verse's half; whether the right or the left half.
\end{definition}



\begin{definition}\label{def:poem}
  \textbf{Poem} \hfill \\
  is a set of verses has the same meter and rhyme.  

\end{definition}

\begin{definition}\label{def:aruds}
  \textbf{al-arud\textarabic{العروض}}\footnote{it is often called the \textit{Knowledge of Poetry}.}; \hfill \\
 it is the study of poetic meters, in which he has laid down rigorous
rules and measures, with them we can determine whether a meter of a poem is sound
or broken. A meter is an ordered sequence of \textit{feet}. Feet are the basic
units of meters, there are eight of them. A Foot consists of a sequence of
consonant and vowels. Traditionally, feet are represented by mnemonic words
called \textit{tafa'il} (\textarabic{تفاعيل}).  According to \textit{al-Farahidi}
and his student, there are sixteen combinations of \textit{tafa'il}. A meter
appears in a \textit{verse} twice; each \textit{shatr} carries the same complete
meter.%


\end{definition}




% NOTE: transpose it. \small{}
\begin{table}[!t]
	\centering
	\begin{tabular}{|c|c|}
		\hline
		\textbf{Feet} & \textbf{Scansion} \\
		\hline
		\textarabic{فَعُولُنْ}  & \texttt{0/0//}\\
		\textarabic{فَاعِلُنْ}  & \texttt{0//0/}\\
		\textarabic{مُسْتَفْعِلُنْ}& \texttt{0//0/0/}\\
		\textarabic{مَفاعِيلُنْ}& \texttt{0/0/0//}\\
		\textarabic{مَفْعُولاَت} & \texttt{0//0///}\\
		\textarabic{فَاعِلاَتُنْ} & \texttt{0/0//0/}\\
		\textarabic{مُفَاعَلَتُنْ}& \texttt{0///0//}\\
		\textarabic{مُتَفَاعِلُنْ}& \texttt{0//0///}\\
		\hline
	\end{tabular}
	\caption{The eight feet. Every digit represents the corresponding diacritic
		over each latter in the feet. \texttt{/} If a letter has got \textit{harakat} (
		\textarabic{◌َ} \textarabic{◌ُ} \textarabic{◌ِ}), \texttt{0} if a letter has got
		\textit{sukun} (\textarabic{◌ْ}). Any \textit{mad} (\textarabic{و, ا, ى}) is
		equivalent to \textit{sukun}.}\label{arud:feet}
\end{table}


For example, the following \textit{shatr} \textarabic{وَيُسأَلُ في الحَوادِثِ ذو صَوابٍ} is
equivalent to the \textit{meter} \textarabic{مفاعلتن مفاعلتن فعول}, which means
it belongs to \textit{Al-Wafeer} meter. We can get the pattern of the
\textit{sukun} and \textit{harakat} by replacing each feet by the corresponding
code in table \ref{arud:feet}, which produces the following pattern that should
be read from right to left:%
\begin{flushright}
	{\texttt{0/0// 0///0// 0///0//}} % You have to filp it.
\end{flushright}
This is a very brief introduction to \textit{Arud}, many details are reduced.




\begin{table}[!t]
	\centering
	\begin{tabular}[h!]{|c|c|}
		\hline
		\textbf{Meter Name} & \textbf{Meter} \small{\textit{feet combination}} \\
		\hline
		\textit{al-Wafeer}    & \textarabic{مُفَاعَلَتُن مُفَاعَلَتُن فَعُولُن} \\ %
		\textit{al-Taweel}    & \textarabic{فَعُوْلُنْ مَفَاْعِيْلُنْ فَعُوْلُنْ مَفَاْعِلُنْ} \\ %
		\textit{al-Taweel}    & \textarabic{فَعُوْلُنْ مَفَاْعِيْلُنْ فَعُوْلُنْ مَفَاْعِلُنْ} \\ %
		\textit{al-Kamel}     & \textarabic{مُتَفَاْعِلُنْ مُتَفَاْعِلُنْ مُتَفَاْعِلُنْ} \\%
		\textit{al-Baseet}    & \textarabic{مُسْتَفْعِلُنْ فَاْعِلُنْ مُسْتَفْعِلُنْ فَاْعِلُنْ} \\%
		\textit{al-Khafeef}   & \textarabic{فَاْعِلاتُنْ مُسْتَفْعِلُنْ فَاْعِلاتُنْ} \\ %
		\textit{al-Rigz}      & \textarabic{مُسْتَفْعِلُنْ مُسْتَفْعِلُنْ مُسْتَفْعِلُنْ} \\%
		\textit{al-Raml}      & \textarabic{فَاْعِلاتُنْ فَاْعِلاتُنْ فَاْعِلاتُنْ} \\ %
		\textit{al-Motakarib} & \textarabic{فَعُوْلُنْ فَعُوْلُنْ فَعُوْلُنْ فَعُوْلُنْ} \\%
		\textit{al-Sar'e}     & \textarabic{مُسْتَفْعِلُنْ مُسْتَفْعِلُنْ مَفْعُوْلاتُ} \\%
		\textit{al-Monsafeh}  & \textarabic{مُسْتَفْعِلُنْ مَفْعُوْلاتُ مُسْتَفْعِلُنْ} \\
		\textit{al-Mogtath}   & \textarabic{مُسْتَفْعِلُنْ فَاْعِلاتُنْ فَاْعِلاتُنْ} \\
		\textit{al-Madeed}    & \textarabic{فَاْعِلاتُنْ فَاْعِلُنْ فَاْعِلاتُنْ } \\
		\textit{al-Hazg}      & \textarabic{مَفَاْعِيْلُنْ مَفَاْعِيْلُنْ} \\%
		\textit{al-Motadarik} & \textarabic{فَاْعِلُنْ فَاْعِلُنْ فَاْعِلُنْ فَاْعِلُنْ} \\%
		\textit{al-Moktadib}  & \textarabic{مَفْعُوْلاتُ مُسْتَفْعِلُنْ مُسْتَفْعِلُن} \\
		\textit{al-Modar'e}   & \textarabic{مَفَاْعِيْلُنْ فَاْعِلاتُنْ مَفَاْعِيْلُنْ} \\				
%		\vdots & \vdots\\
		\textit{al-Kamel}     & \textarabic{مُتَفَاْعِلُنْ مُتَفَاْعِلُنْ مُتَفَاْعِلُنْ} \\%
		\textit{al-Baseet}    & \textarabic{مُسْتَفْعِلُنْ فَاْعِلُنْ مُسْتَفْعِلُنْ فَاْعِلُنْ} \\%
		\hline
	\end{tabular}
	\caption{\textit{The sixteen Arabic poem meters}}\label{arud:meters}
\end{table}
\bigskip
\newpage

\section{Thesis Objectives}





%\begin{figure}
%	
%	\input{MeterSizesBarChart}
%	
%	\caption{Meter names are on the $x$-axis, size  is on the $y$-axis.}
%	\label{data_size}
%\end{figure}



