\chapter{\uppercase{Results And Discussion}}\label{ch_results}

In this chapter we will explain the results of all the 192 experiments on our dataset. We measure the results using the overall $F_1$ Score Then we measure the performance accuracy of the model per class (meter).We will start by present the results for every combinations and then discuss our findings related to the topic.

\section{Results}

As we explained, In Chapter~\ref{ch_model_training} we have a set of combinations we need to explore it. So, most of our results will combine a combination and show the results of this combinations. Let's explore it as below,
\begin{enumerate}
\item  We have three data representation \textbf{\textit{Binary, One-hot, and Two-Hot}} we will represent it as \textbf{\textit{BinE, 1D, 0T}} respectively.
\item We have two types of model loss functions \textbf{\textit{Weighting loss and no Weighting loss}} we will represent it as \textbf{\textit{(1 and 0)}} respectively.
\item Number of layers is represented as \textbf{\textit{nL}} for example, 7 layers is 7L.
  \item Number of cell units is represented as \textbf{\textit{nU}} for example, 82 unit is 82U.
  \end{enumerate}

  So, If we need to explain a set of combination we can write (4L, 82U, 0) which means 4 layers, 82 units, and no weighted loss function. Also, we will provide many figures every figure will explain specific result perspective.
  
\subsection{Overall F1 Score}

\textbf{$F_1$ (also F-score or F-measure)} is a measure of a test's accuracy. It considers both the precision $p$ and the recall $r$ of the test to compute the score: $p$ is the number of correct positive results divided by the number of all positive results returned by the classifier, and $r$ is the number of correct positive results divided by the number of all relevant samples (all samples that should have been identified as positive).\\ The F1 score is the harmonic average of the precision and recall, where an F1 score reaches its best value at 1 (perfect precision and recall) and worst at 0~\footnote{https://en.wikipedia.org/wiki/F1\_scorex}.
\newpage

We present the $F_!$ Score in Figure~\ref{fig:ArabicModelsResults} of the 16 neural netwroks configurations and at each of the 12 data representations (y- and
x-axis respectively). The x-axis is divided into 4 strips corresponding to the 4 combinations of trimming and diacritic parameters. Then, each strip includes the 3 different encoding values. Each point on the figure represents the F1 score of one of the 192 experiments; (some values are too small, and hence omitted from the figure). To explain the figure, we take as an example the most-left
vertical list of points that represents the 16 experiments of the full (no trimming), diacritics, and binary encoding dataset representation.

The best network out of the 16 configurations is listed at the top: 7 layers, size of 82, and no loss weighting (7L, 82U, 0W). This network possess 90.25 F1 score by the Bi-LSTM cell (indicated by the large circle).

The best LSTM model is indicated by the square point, and possess 79.77 F1 score. Among all the 192 experiments, the highest F1 score is 96.38 and is possessed by a network configuration of (4L, 82U, 0) on (1T, 0D, BinE).

\subsubsection{Data Representation Effects}

In this section we will explain the effect of the 12 data representation technique we explained it previous.

\begin{enumerate}
\item \textbf{Trimming Effect:} The effect of trimming( remove the small classes from training cycle) can be observed if we fix the other two parameters, diacritic and encoding. The score with trimming is consistently higher than that with no trimming. E.g., by looking at the two unshaded strips, the score at (1T, 0D, TwoE) is 0.9629, while that at (0T, 0D, TwoE) is 0.9411. The only exception, with a very little difference, is (1T, 1D, BinE) vs. (0T, 1D, BinE). We need highlight that is logic to have this effect as the training will have less classes with huge amount of data for these classes.
\item \textbf{Diacritics Effect}
  \begin{itemize}
  \item \textit{Without Trimming:} The effect of diacritics is obvious only with no trimming (the two left strips of the figure), where, for each encoding, the F1 score is higher for diacritics than no diacritics.
    \item \textit{With Trimming:} The diacritics doesn't have except for the \textit{one-hot} encoding but other encoding doesn't have an effect on the model performance. This result is inconsistent with what is anticipated from the effect of diacritics. We think that this result is an artifact due to the small number of network configurations.

    \end{itemize} 
\item \textbf{Encoding Effect:} The effect of encoding is clear; by looking at each individual strip, \textit{$F_1$ score is consistently highest for two-hot then one-hot then binary—the} only exception is (1T, 0D, BinE) that performs better than the other two encodings. It seems that two-hot encoding makes it easier for networks to capture the patterns in data. However; we anticipate that there is a particular network architecture for each encoding that is capable of capturing the same pattern with yielding the same score.

  
  \end{enumerate}


  
%\begin{figure}
 % \begin{tikzpicture}%[scale=1.5]
  %% Uses pointModelsFiguresStyle macro defined in pgfplot_configurations.tex

  %% two grid to help during positioing.
  % \draw[step=0.2, green!40, thin] (0,0) grid (8,6);
  % \draw[step=1, red!40, very thin] (0,-1) grid (8,7);

  %% Variables
  \def \maxHeight{5.7}
  \def \yONE{-0.8}
  \def \yTWO{-0.3}

  %% two colored areas to distinguish the diacritics models.
  \fill[gray!30, opacity=0.2, rounded corners=2pt] (0,-0.5) rectangle (2.1,\maxHeight);
  \fill[gray!30, opacity=0.2, rounded corners=2pt] (4.2,-0.5) rectangle (6.32,\maxHeight);

  %% The Full/Elimenated Seperator
  \draw[dashed, thick] (4.2, 0 - 0.9) -- (4.2, \maxHeight +0.5);

  %% Group Labels Full/Elimenated
  \node [align=center, text width=3cm, inner sep=0.25cm] at (2.1, \yONE) {\scriptsize No Trimming(0T)};
  \node [align=center, text width=4cm, inner sep=0.25cm] at (6.32,\yONE) {\scriptsize Trimming(1T)};

  \node [align=center, text width=3cm, inner sep=0.25cm] at (1, \yTWO) {\scriptsize Diacritics(1D)};
  \node [align=center, text width=3cm, inner sep=0.25cm] at (3, \yTWO) {\scriptsize No Diacritics(0D)};
  \node [align=center, text width=3cm, inner sep=0.25cm] at (5+0.2, \yTWO) {\scriptsize Diacritics(1D)};
  \node [align=center, text width=3cm, inner sep=0.25cm] at (7+0.2, \yTWO) {\scriptsize No Diacritics(0D)};

  % Points annotaions
  \def \layerHeight{5.7}
  \def \unitHeight{\layerHeight - 0.3}
  \def \weightedHeihgt{\layerHeight - 0.6}

  \def \step{0.2}

  %%%%%% 

  \node  at (3.5*   \step -0.15, \layerHeight) {\scriptsize 7L};
  \node  at (3.5*   \step -0.15, \unitHeight) {\scriptsize 82U};
  \node  at (3.5*   \step -0.1, \weightedHeihgt) {\scriptsize 0};

  \node  at (5.2*   \step, \layerHeight) {\scriptsize 7L};
  \node  at (5.2*   \step, \unitHeight) {\scriptsize 50U};
  \node  at (5.2*   \step, \weightedHeihgt) {\scriptsize 1};

  \node  at (7*     \step +0.15, \layerHeight) {\scriptsize 7L};
  \node  at (7*   \step   +0.15, \unitHeight) {\scriptsize 50U};
  \node  at (7*   \step   +0.1, \weightedHeihgt) {\scriptsize 1};

  %%%%%%

  \node  at (14*    \step -0.15, \layerHeight) {\scriptsize 7L};
  \node  at (14*   \step -0.15, \unitHeight) {\scriptsize 82U};
  \node  at (14*   \step -0.1, \weightedHeihgt) {\scriptsize 0};

  \node  at (16*    \step, \layerHeight) {\scriptsize 7L};
  \node  at (16*    \step, \unitHeight) {\scriptsize 82U};
  \node  at (16*    \step, \weightedHeihgt) {\scriptsize 0};

  \node  at (17.6*  \step +0.15, \layerHeight) {\scriptsize 7L};
  \node  at (17.6*  \step +0.15, \unitHeight) {\scriptsize 50U};
  \node  at (17.6*  \step +0.1, \weightedHeihgt) {\scriptsize 0};

  %%%%%%

  \node  at (24.5*  \step -0.15, \layerHeight) {\scriptsize 7L};
  \node  at (24.5*  \step -0.15, \unitHeight) {\scriptsize 82U};
  \node  at (24.5*  \step -0.1,  \weightedHeihgt) {\scriptsize 1};

  \node  at (26.1*  \step, \layerHeight) {\scriptsize 7L};
  \node  at (26.1*  \step, \unitHeight) {\scriptsize 82U};
  \node  at (26.1*  \step, \weightedHeihgt) {\scriptsize 0};

  \node  at (28*    \step +0.15, \layerHeight) {\scriptsize 7L};
  \node  at (28*    \step +0.15, \unitHeight) {\scriptsize 82U};
  \node  at (28*    \step +0.1,  \weightedHeihgt) {\scriptsize 1};

  %%%%%%

  \node  at (35*    \step -0.15, \layerHeight) {\scriptsize 4L};
  \node  at (35*    \step -0.15, \unitHeight) {\scriptsize 82U};
  \node  at (35*    \step -0.1,  \weightedHeihgt) {\scriptsize 0};

  \node  at (37*    \step, \layerHeight) {\scriptsize 7L};
  \node  at (37*    \step, \unitHeight) {\scriptsize 50U};
  \node  at (37*    \step, \weightedHeihgt) {\scriptsize 0};

  \node  at (38.7*  \step +0.15, \layerHeight) {\scriptsize 4L};
  \node  at (38.7*  \step +0.15, \unitHeight) {\scriptsize 50U};
  \node  at (38.7*  \step +0.1,  \weightedHeihgt) {\scriptsize 1};

  \begin{axis}[
    major x tick style = transparent,
    ybar=2*\pgflinewidth,
    x=10pt,
    ymajorgrids = true,
    % every axis y label/.style= {at={( 0, 1.09)}, anchor=north},
    ylabel = {$F_1$ Score},
    ylabel style = {font=\footnotesize},
    xtickmin={1},
    xtickmax={21},
    axis x line = bottom,
    axis y line = left,
    enlarge y limits={upper, value=0.1},
    xticklabels = {,,},
    bar shift=0pt,
    % scaled y ticks = false,
    enlarge x limits=0.1,
    ymin=0.76,
    ymax=0.98,
    legend style={at={(0.5, -0.2)}, anchor=north, legend columns=3},
    % legend style={at={(0.9, 0.26)}, anchor=north, legend columns=1},
    every axis legend/.append style={nodes={right}},
    nodes near coords={\vspace*{0.1\baselineskip}
      \foreach \X in \pgfplotspointmeta%
      {\centerline{\X}\newline}%
      \vspace*{-0.7\baselineskip}
    },
    nodes near coords style={font=\scriptsize,anchor=-90, text width=1cm},
    ]

    % Settings:
    % x-axis values are 1,2,3,..,21

    % Full takes 1,..,10
    % Full-with-diacritics takes 1,2,3
    % Full-without-diacritics takes 7,8,9

    % (14,21
    % (14, 15
    % (14, 21

    % Separators 5, 11, 17

    %%%%%%%%%%%%%%%
    %% Full Dataset
    %%%%%%%%%%%%%%%
    % Binary
    \addplot[pointBiLSTM=blue] coordinates {
      (1, 0.9025) %[7L~~,82U~~~,0] % Full,Diacritics,Binary
      (7, 0.8767)  %[7L~~~,82U~~~,0] % Full,Without-Diacritics,Binary

    };
    % OneHot
    \addplot[pointBiLSTM=red] coordinates {
      (2, 0.9347) %[7L,50U,1] % Full,Diacritics,OneHot
      (8, 0.9310) %[7L,82U,0] % Full,Without-Diacritics,OneHot
    };
    % TwoHot
    \addplot[pointBiLSTM=brown] coordinates {
      (3, 0.9485) %[~7L,~~~50U,1] % Full,Diacritics,TwoHot
      (9, 0.9411) %[7L,50U,0] % Full,Without-Diacritics,TwoHot
    };
    %%%%%%%%%%%%%%%%%%%%%
    %% Trimmed Dataset
    %%%%%%%%%%%%%%%%%%%%%
    % Binary
    \addplot[pointBiLSTM=blue] coordinates {
      (13, 0.8989)  %[7L~~,82U~~~,1]  % (14,Binary
      (19, 0.9638)  %[4L,82U,0]  % (14,Binary
    };
    % OneHot
    \addplot[pointBiLSTM=red] coordinates {
      (14, 0.9473) %[7L~~,82U~~~,0] % (14,OneHot
      (20, 0.9435) %[7L,50U,0] % (14,OneHot
    };
    % TwoHot
    \addplot[pointBiLSTM=brown] coordinates {
      (15, 0.9547) %[~7L,~~~82U,1] % (14,TwoHot
      (21, 0.9629) %[3L,50U,1] % (14,TwoHot
    };
    %%%%%%%%%%%%%%%%%%%%%
    %% LSTM models [squares]
    %%%%%%%%%%%%%%%%%%%%%
    % Binary
    \addplot[pointLSTM=blue] coordinates {
      (1,  0.7977761225792722)
      (7,  0.7883237568276938)
      (13, 0.827841810615813)
      (19, 0.8163877496033292)
    };
    % OneHot
    \addplot[pointLSTM=red] coordinates {
      (2,  0.8965678276701898)
      (8,  0.8869854106074577)
      (14, 0.9094811843247612)
      (20, 0.926671257072718)
    };
    % TwoHowt
    \addplot[pointLSTM=brown] coordinates {
      (3,  0.9170331749071908)
      (9,  0.8972239956491925)
      (15, 0.94120887345448)
      (21, 0.9284854653773612)
    };
    %%%%%%%%%%%%%%%%%%%%%
    %% The rest models
    %%%%%%%%%%%%%%%%%%%%%
    % Binary
    \addplot[pointRug=blue, mark size=0.7pt]
    coordinates {
      % 1, 7, 13, 19
      (19, 0.8182558452833577)
      (19, 0.9638475586025208)
      (19, 0.9619555129778764)
      (19, 0.9611891147501721)
      (19, 0.6499715594407688)
      (19, 0.9576864353501182)
      (19, 0.7611831272640182)
      (19, 0.6869083615244139)
      (19, 0.7495314792084543)
      (19, 0.6293147322096818)
      (19, 0.704074484327755 )
      (19, 0.9556686525162408)
      (19, 0.9604047540640064)
      (19, 0.9621052001317248)
      (19, 0.7838038499535971)

      (13, 0.8986558093584407)
      (13, 0.8988593838876747)
      (13, 0.8555698589947011)
      (13, 0.8649043499086908)
      (13, 0.8203873903541598)
      (13, 0.7581235218393557)
      (13, 0.8634134658563603)
      (13, 0.6457563691883963)
      (13, 0.7879471903721222)
      (13, 0.7406280872975481)
      (13, 0.8841181929766787)
      (13, 0.8741370535580636)
      (13, 0.770906804778014 )
      (13, 0.8288896206927522)
      (13, 0.8079693440708918)

      (7, 0.2392116525974793)
      (7, 0.7722328627840438)
      (7, 0.817325199214963 )
      (7, 0.8267420964271358)
      (7, 0.8307795984961339)
      (7, 0.2392116525974793)
      (7, 0.7268331323449433)
      (7, 0.7918174079591402)
      (7, 0.8564115296398761)
      (7, 0.7346362110141638)
      (7, 0.2392116525974793)
      (7, 0.7977524768863352)
      (7, 0.8766640656404436)
      (7, 0.5082878153744296)
      (7, 0.7272055520087016)

      (1, 0.8614953536213379)
      (1, 0.8794601688302476)
      (1, 0.2392116525974793)
      (1, 0.2392116525974793)
      (1, 0.8706225910950319)
      (1, 0.8677259937102457)
      (1, 0.6374642358894327)
      (1, 0.5386134165661725)
      (1, 0.2392116525974793)
      (1, 0.2392116525974793)
      (1, 0.8948003121231468)
      (1, 0.6666016410110899)
      (1, 0.866862925918044 )
      (1, 0.9025442765600246)
      (1, 0.8869558534912866)
    };

    % OneHot
    \addplot[pointRug=red, mark size=0.7pt]
    coordinates {
      % 2, 8, 14, 20

      (20, 0.9234859144388228 )
      (20, 0.24334939975451306)
      (20, 0.24334939975451306)
      (20, 0.24334939975451306)
      (20, 0.9015956650600244 )
      (20, 0.24334939975451306)
      (20, 0.24334939975451306)
      (20, 0.24334939975451306)
      (20, 0.9361973475436338 )
      (20, 0.9426518576175792 )
      (20, 0.9435140556237464 )
      (20, 0.24334939975451306)
      (20, 0.9382570427805884 )
      (20, 0.9427656198545042 )
      (20, 0.9328263928389664 )

      (14, 0.9334371164266684 )
      (14, 0.24334939975451306)
      (14, 0.93929287788522   )
      (14, 0.9401011885160016 )
      (14, 0.940059276112924  )
      (14, 0.24334939975451306)
      (14, 0.24334939975451306)
      (14, 0.24334939975451306)
      (14, 0.24431937251145106)
      (14, 0.93308984222974   )
      (14, 0.9352632997036194 )
      (14, 0.8778253450288896 )
      (14, 0.24334939975451306)
      (14, 0.24334939975451306)
      (14, 0.9472562344699575 )


      (8, 0.9298964318649358)
      (8, 0.2392116525974793)
      (8, 0.9212184625570452)
      (8, 0.2392116525974793)
      (8, 0.2392116525974793)
      (8, 0.2392116525974793)
      (8, 0.2392116525974793)
      (8, 0.2392116525974793)
      (8, 0.2392116525974793)
      (8, 0.2392116525974793)
      (8, 0.2392116525974793)
      (8, 0.9026743278711784)
      (8, 0.9146094913811448)
      (8, 0.9298964318649358)
      (8, 0.9310432479723818)
      (8, 0.9275673311106382)

      (2, 0.2392116525974793)
      (2, 0.9346846846846848)
      (2, 0.9320954813080796)
      (2, 0.922241138776572 )
      (2, 0.8536213378733064)
      (2, 0.2392116525974793)
      (2, 0.9248953678087536)
      (2, 0.2392116525974793)
      (2, 0.2392116525974793)
      (2, 0.2392116525974793)
      (2, 0.2392116525974793)
      (2, 0.9309841337400392)
      (2, 0.2392116525974793)
      (2, 0.9336324513489868)
      (2, 0.2392116525974793)
    };
    % TwoHot
    \addplot[pointRug=brown, mark size=0.7pt]
    coordinates {
      % 3, 9, 15, 21

      (21, 0.9100978953986169 )
      (21, 0.24334939975451306)
      (21, 0.24334939975451306)
      (21, 0.24334939975451306)
      (21, 0.9445379157560698 )
      (21, 0.9396281771098404 )
      (21, 0.9380534682513548 )
      (21, 0.24334939975451306)
      (21, 0.9627219112055804 )
      (21, 0.9628835733317368 )
      (21, 0.9255635721342396 )
      (21, 0.9427356824237344 )
      (21, 0.24334939975451306)
      (21, 0.959949705116307  )
      (21, 0.24334939975451306)

      (15, 0.24334939975451306)
      (15, 0.9455737508607012 )
      (15, 0.9467353231745652 )
      (15, 0.9530221836362004 )
      (15, 0.9441487291560638 )
      (15, 0.8771727090381104 )
      (15, 0.8684369667395144 )
      (15, 0.9522318354638803 )
      (15, 0.9250725982696164 )
      (15, 0.24334939975451306)
      (15, 0.9523276352423437 )
      (15, 0.9143310481094512 )
      (15, 0.9537646319192888 )
      (15, 0.9335987785528246 )
      (15, 0.9546926922731492 )

      (9, 0.9329644605235156)
      (9, 0.9195396183585159)
      (9, 0.935766475136554 )
      (9, 0.2392116525974793)
      (9, 0.9373034451774608)
      (9, 0.2392116525974793)
      (9, 0.9307122082712634)
      (9, 0.2392116525974793)
      (9, 0.2392116525974793)
      (9, 0.2392116525974793)
      (9, 0.9339575796268708)
      (9, 0.94113404743326  )
      (9, 0.2392116525974793)
      (9, 0.2392116525974793)
      (9, 0.2392116525974793)

      (3, 0.2392116525974793)
      (3, 0.2392116525974793)
      (3, 0.9363812631529168)
      (3, 0.2392116525974793)
      (3, 0.9446217871414722)
      (3, 0.9440129105483436)
      (3, 0.9124399990541724)
      (3, 0.9427892459388524)
      (3, 0.9458158946347924)
      (3, 0.2392116525974793)
      (3, 0.2392116525974793)
      (3, 0.9424582062377338)
      (3, 0.9417192783334516)
      (3, 0.2392116525974793)
      (3, 0.2392116525974793)
    };

    \legend{{\scriptsize{Binary(BinE)}}, {\scriptsize{One-Hot(OneE)}}, {\scriptsize{Two-Hot(TwoE)}}}
  \end{axis}
\end{tikzpicture}


%%% Local Variables:
%%% mode: latex
%%% TeX-master: "../../Thesis.tex"
%%% End:
 % \caption{$F_1$ score of the 192 experiments plotted as 12 vertical rug plots (for the 12 different data representations: $\left\{\mathit{Trimming},\ \mathit{No Trimming} \right\} \times \left\{\mathit{Diacritics},\ \mathit{No Diacritics} \right\} \times \left\{\mathit{OneE},\ \mathit{BinE},\ \mathit{TwoE}\right\}$), each represents 16 exp. (for the 16 different network configurations: $\left\{7L,\ 4L\right\} \times \left\{82U,\ 50U\right\} \times \left\{0W, 1W\right\} \times \left\{LSTM,\ BiLSTM\right\}$). For each rug plot the two best (Bi)LSTM models are marked differently; and the other three network configuration parameters of the best of them, which consistently was the BiLSTM, are listed at the top of each rug plot.}~\label{fig:ArabicModelsResults}
%\end{figure}

  \newpage
\subsubsection{Network Configurations Effects} 

This section is to comment on the effect of the network configurations parameters.
\begin{itemize}
\item \textbf{Cell Type}: It is clear that BI-LSTM  (large circle) is the highest $F_1$ score for each data representation. It always higher than the highest score of the LSTM model (large square). This is what we expected the more complex architecture, the more results we can achieved. But we need to mention that the BI-LSTM is slower than LSTM in overall running time for all experiments, and it also consume much more resources than LSTM cell.
\item \textbf{Layers Number:} As we explained in Section~\ref{sec_deep_learning_background} The idea behind the deep neural network come from the multi-layers which makes the network learn more details. So, the more complex network (more layer) the more results we can achieved. So, in our experiments we can show that 7 layers achieved the highest scores more than the 4 layers. There is exception for the trimming data without diacritics in (1T, 0D, BinE) and (1T, 0D, TwoE). The straightforward interpretation for that is the reduction in dataset size occurred by trimming and no diacritics, which required less complex network. So, of we reduced the complexity of our problem the number of layers will not be effective.

\item \textbf{Cell Units and Weighting Loss:} We can't figure out a consistent effect based on the number of cell units or the weighting loss. But we need to mention that the highest results achieved was using both the highest cell units 82 and the weighted loss.
  \end{itemize}



\subsection{Per-Class (Meter) Accuracy}

Next, we investigate the accuracy of each class rather than the overall F1 score. For each of the four combinations of trimming × diacritics, we select the best model. From Figure 4, it is clear that three of them will be of two-hot encoding and the fourth (1T, 0D, BinE) is best overall model discussed above. Figure 5 displays the per-class accuracy of these four models. The class names are ordered according to their individual sizes (the same order of Figure 1). Several comments are in order. The overall score of each of the four models is around 0.95 (Figure 4); however, for the four models the per-class score of only 6 classes are around this value. For some classes the score drops significantly.

Moreover, the common trend for the four models is that the per-class accuracy decreases with the class size for the first 11 classes. The score of the two Trimmed models keeps decreasing significantly with the remaining 5 classes. Although this trend is associated with class size, the latter needs not be the cause of this phenomenon. This phenomenon, along with what was concluded above for the inconsistent effect of the weighting loss, emphasizes the importance of a more prudent design for the weighting function in combination with appropriate selection of the both the learning rate and batch size. Also, the same full set of experiments can be reconducted with enforcing all classes to have equal size to assert/negate the interpretation of causality (Sec. VI).

\subsection{Encoding Effect}

Encoding Effect on Learning rate and Memory Utilization: Figure 6-a shows the learning curve of the best model (4L, 82U, 0W, 1T, 0D, BinE), which converges to the same value displayed in Figure 4, along with its learning curve using the other two encodings. There is no big difference in convergence speed except normal few epochs for the curve possessing higher convergence score.

Figure 6-b is a sort of visual emphasis, in bytes, on the same numbers of Figure 2 that represented the size of a character vector under each encoding—(1,448, 328, 64) = 8 × (8, 41,181). It is clear that one-hot encoding is very memory demanding than the two other encodings. Twohot is a good compromise between the one-hot that is too memory demanding and the binary that is too terse, and hence difficult for networks to learn from (as discussed above for encoding effect and will be elaborated more in Sec. VI).

