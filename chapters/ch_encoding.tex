\chapter{\uppercase{Data Encoding}}\label{ch:data_encoding}

As we explained, We have collected the dataset and cleaned the data from any quality issues. The next step is to change the data representation to be ready for model training. This change of the data structure named \textit{Data Encoding}.

\section{Work embedding Encoding in English}\label{sec:word-level-english}
The concept of data encoding was first introduced by [Bengio et al., 2003]~\cite{Bengio2003}. They used an embedding lookup table as a reference and map every word to this lookup. They used the resulting dense vectors as input for language modeling. There are many works to improve the word embedding one of them [Collobert et al., 2011]~\cite{Collobert_2011} proposed improvement of word embedding task and proved the versatility of word embedding in many NLP tasks. Another work proposed by [Mikolov et al., 2013~\cite{Mikolov_2013};
Jeffrey Pennington et al., 2014~\cite{Pennington_2014} ] shows the maturity of word embedding and is currently the most used encoding technique in the neural network based natural language processing.

\section{Character Level Encoding in English}\label{sec:char-level-english}

All the previous work focused on word embedding encoding, but in our research problem here we do not work on word level we focus into character level encoding as input feature to the model. There is a good deal of research based on the character level encoding [Kim et al., 2015]~\cite{Kim_2015} used character level embedding to construct word level representations to work on out of vocabulary problem. [Chiu and Nichols, 2015]~\cite{Chiu_2015} also used character embeddings with a convolutional neural network for named entity recognition.[Lee, Jinhyuk et al.,2017]~\cite{ijcai_2017} used character embeddings for the personal name classification using Recurrent Neural Networks.

\section{Character Level Encoding in Arabic}\label{sec:char-level-arabic}

Working on Arabic language embedding based on the character level did not take much attention from the research community. [Potdar et al.,2017]~\cite{Potdar_2017} has done a comparative study on six encoding techniques. We are interested in the comparison of one-hot and binary. They have used Artificial Neural Network for evaluating cars based on seven ordered qualitative features. The accuracy of the model was the same in both encoding one-hot and binary. [Agirrezabal et al.,2017]\cite{Agirrezabal_2017} shows that representations of data learned from character-based neural models are more informative than the ones from hand-crafted features.

In this research, We will make a comparative study of different encoding techniques between binary and one-hot. Also, we provide some new encoding method specific for Arabic letters, and we will see the effect of this on our problem. We will show the efficiency of every technique based on performing model training and model running time performance.

Generally, a character will be represented as an n vector. Consequently, a verse would be an $n \times p$ matrix, where n
is the character representation length and p is the verse’s length, n varies from one encoding to another, we have used One-Hot and Binary encoding techniques and proposed a new encoding, the \textbf{Two-Hot} encoding.

Arabic letters have a feature related to the diacritics; To explain this feature we will take an example based on \textit{One-Hot} encoding. This feature is related to how we will treat the character with a diacritic. Arabic letters are 36 + white space as a letter. So, the total is 37. Any letter represented as a vector $37 \times 1$. Let's take an example a work such as \textarabic{مرحبا} having 5 letters encoded as a $37 \times 5$ matrix. If it came with diacritics such as \textarabic{مَرْحَبَا} and we need to represent the letters as One-Hot encoding we will consider every letter and diacritics as a separate letter. So, it will be 5 character and 4 diacritics. The vector shape will be $41 \times 9$.

One of the main reason we need to care about the encoding is the  \textit{RNN} training. If we have a different number of time steps in  \textit{RNN} cell and the input vector dimensions are different based on the input, It will have a standard architecture for the model and to be able to train both the work with diacritics and without diacritics to show the effect of the model learning on the same architecture.

To achieve the model architecture unification,  we proposed three different encoding systems: \textit{one-hot}, \textit{binary}, and the novel encoding system developed in this project \textit{two-hot}. The three of them explained in the next three subsections.

\begin{figure*}
  \centering
    \begin{tikzpicture}
    \node at (-12,0){\input{./Figures/Ch_5_Encoding/encoding_onehot.tex}};
    \node at (-7,0){\input{./Figures/Ch_5_Encoding/encoding_binary}};
    \node at (0,0){\input{./Figures/Ch_5_Encoding/encoding_twohot.tex}};
    \node at (-12+0.3,-3){(a) \textit{One-Hot}};
    \node at (-7+0.3, -3){(b) \textit{Binary}};
    \node at (0-0.5,  -3){(c) \textit{Two-Hot}};
\end{tikzpicture}


\caption{Different encoding mechanisms%: \textit{One-hot}, \textit{binary}, and \textit{two-hot}  encoding. Using the word example %\textarabic{مَرْحَبَا} \textit{One-hot} encoding is applied using $n$-letter alphabet $n = 181$ in Arabic. Same word encoded using \textit{binary} encoding where the vector length $n = \ceil*{\log_2 l}$, $l \in \{181\}$. \textit{two-hot} encoding is applied by stacking $\bm k_{4 \times 1}$ on the top of $\bm m_{37 \times 1}$ we get the $Two-Hot_{41 \times 1}$ \usebox{\columnVector}, which represents a letter and its diacritic simultaneously. 
}
\label{fig:One-Binary-Encoding}
\end{figure*}

\newpage
\subsection{One-Hot encoding}\label{sec:one-hot-encoding}
In this encoding system, We assume the letter with the diacritic as one unit. So, for example, \textarabic{د} represented as letter differs than (\textarabic{دَ, دِ, دُ, دْْ}). Now every letter is represented 5 times one without diacritic and four times with different diacritics combinations $36 \times 5$ besides the white-space character. So, the total is $181 \times 1$. From now forward, We have 181-characters Arabic alphabet represent the One-Hot encoding, and according to it, we will encode verses. (Figure~\ref{fig:One-Binary-Encoding}).

We need to mention that, One-Hot encoding technique is of the famous techniques in encoding problem. We will not compare the encoding technique in these sections. However, We will discuss it in details in model results. Also, The implementations of the One-Hot is trivial. But we need to focus on here about the size for every letter which is $181 \times 1$ which means if we have a verse with 82 characters it will results up with a matrix $181 \times 82$ which is very big to be in memory.

\subsection{Binary Encoding}\label{sec:binary-encoding}

The idea is to represent a letter with an $n \times 1$ vector which contains a unique combination of ones and zeros.  $n =\ceil*{\log_2} l$ where $l$ is the alphabet length, and $n$ is the sufficient number of digits to represent $l$ unique binary combinations.  For example a phrase like this \textarabic{مَرْحَبا}, it has 5 characters, figure~\ref{fig:One-Binary-Encoding} shows how it is encoded as a $8 \times 5$ matrix, which saves 22.6 times memory than the \textit{one-hot} and reduces the model input dimensions, significantly. But on the other hand, the input letter share some features between them due to the binary representation as it is shown in figure~\ref{fig:One-Binary-Encoding}.

\subsection{Two-Hot encoding}\label{sec:two-hot-encoding}
This is an intermediate technique which takes the advantages of the previous two encoding techniques. In which we encode the letter and its diacritic separately as two \textit{One-Hot} vectors, this way the letter is encoded as $37 \times 1$ \textit{One-Hot} vector and the diacritic is encoded as $4 \times 1$ \textit{One-Hot} vector, then both vectors stacked to form one $41 \times 1$ vector (Figure~\ref{fig:One-Binary-Encoding}).

By this way, we reduced the vector dimension from 181 to 41 and also minimizes the number of shared features between vectors to the maximum one at each vector. 


