\addcontentsline{toc}{chapter}{References}
\makeatletter
\renewcommand{\ps@plain}{%
\renewcommand\@oddhead{\hfil\normalfont\textrm{\thepage}}%
\renewcommand\@evenhead{}%
\renewcommand\@oddfoot{}%
\renewcommand\@evenfoot{}%
}
\makeatother
\pagestyle{myheadings}
\renewcommand\bibname{\uppercase{References}}
\begin{thebibliography}{999}
\bibitem{Alnagdawi2013}
\bibitem{Almuhareb2015}Abdulrahman Almuhareb
\bibitem{Alkafi1994} Al-Khatib Al tabrisi 1994. Al-Kafi in Al-Arud and Al-Quafi. Al-Khangi Press.
 

    
\bibitem{AlQuaed} القواعد العروضية وأحكام القافية العربية , محمد بن فلاح المطيري
\bibitem{Goodfellow-et-al-2016} Deep Learning,
    author={Ian Goodfellow and Yoshua Bengio and Aaron Courville},
    publisher={MIT Press},
    note={\url{http://www.deeplearningbook.org}},
    year={2016}

    \bibitem{Zeiler2014} Zeiler, M. D. and Fergus, R. (2014). Visualizing and understanding convolutional networks. In ECCV’14.

      \bibitem{Cox2958} Cox, D. R. "The Regression Analysis of Binary Sequences." Journal of the Royal Statistical Society. Series B (Methodological) 20, no. 2 (1958): 215-42. http://www.jstor.org/stable/2983890.

\bibitem{DLFundamentals}Fundamentals of Deep Learning by Nikhil Buduma and Nicholas Locascio (O’Reilly).
\bibitem{colah} Colah, “Understanding Lstm Networks,” 2015. [Online]. Available: http://colah.github.io/posts/2015-08-Understanding-LSTMs/
\bibitem{Mikolov_et_al} Mikolov et al., 2010 Tomas Mikolov, Martin Karafiát,Lukas Burget, Jan Cernockỳ, and Sanjeev Khudanpur. Recurrent neural network based language model. In Inter-speech, volume 2, page 3, 2010.
\bibitem{Bengio_ et_ al} Yoshua Bengio, Patrice Simard, and Paolo Frasconi. Learning long-term dependencies with gradient descent is difficult. IEEE transactions on neural networks, 5(2):157–166, 1994.

\bibitem{Zaremba_et_al} Wojciech Zaremba, Ilya Sutskever, and Oriol Vinyals. Recurrent neural network regularization. arXiv preprint arXiv:1409.2329, 2014

\bibitem{Cho_et_al} Kyunghyun Cho, Bart Van Merriënboer, Dzmitry Bahdanau, and Yoshua Bengio. On the properties of neural machine translation: Encoder-decoder approaches. arXiv preprint arXiv:1409.1259, 2014.

\bibitem{Hochreiter} Sepp Hochreiter and Jürgen Schmidhuber. Long short-term memory. Neural computation, 9(8):1735–1780, 1997.
%Learning Phrase Representations using RNN Encoder–Decoder for Statistical Machine Translation

\bibitem{Abuata2016RuleBasedAlgorithmFor}Abuata, Belal and Al-Omari‏, A,A Rule-Based Algorithm for the Detection of Arud Meter in Classical Arabic Poetry, 2016‏,researchgate.net‏

\bibitem{Alnagdawi2013FindingArabicPoemMeter} Alnagdawi, Mohammad a and Rashideh, Hasan and Fahed, Ala, Finding Arabic Poem Meter Using Context Free Grammar,J. of Commun. {\&} Comput. Eng., 2013, volume  3,1,52-59

\bibitem{Kurt2012AlgorithmForDetectionAnalysis} Kurt, Atakan and Kara, Mehmet, An Algorithm for the Detection and Analysis of Arud Meter in Diwan poetry, Turkish Journal of Electrical Engineering and Computer Sciences, 2012, 20, 6, 948-963

\bibitem{Almuhareb2015RecognitionModernArabicPoems} Almuhareb, Abdulrahman and Almutairi, Waleed A and Altuwaijri, Haya and Almubarak, Abdulelah and Khan, Marwa, Recognition of Modern Arabic Poems, 10, 4, 454--464, 2015

\bibitem{Tizhoosh2006PoemRecognition} Tizhoosh, H. R. and Dara, R. A., On Poem recognition, Pattern Analysis and Applications, 2006, 9, 4, 325-338

\bibitem{Tanasescu2016AutomaticClassificationPoetryMeter} Tanasescu, Margento Chris and Paget, Bryan and Inkpen, Diana, Automatic Classification of Poetry By Meter and Rhyme, Florida Artificial Intelligence Research Society Conference, 2016, 5

\bibitem{diwan} \textarabic{الدّيْوَانُ}, 2013, https://www.aldiwan.net.

\bibitem{PoetryEncyclopedia2016}\textarabic{المَوسُوعَةُ الشِعْرية}, 2016, https://poetry.dctabudhabi.ae

\bibitem{ArabicpoetryDS} W. A. Yousef, O. M. Ibrahime, T. M. Madbouly, M. A. Mahmoud, A. H. El-Kassas, A. O. Hassan, and A. R. Albohy, “Arabic Poem Comprehensive Dataset,” 2018. [Online]. Available: https://hcilab.github.io/ArabicPoetry-1-Private/\#APCD

   \bibitem{Bengio2003} Yoshua Bengio, Réjean Ducharme,
Pascal Vincent, and Christian Jauvin. A neural probabilis-
tic language model. Journal of machine learning research,
3(Feb):1137–1155, 2003.

\bibitem{Collobert_2011} Ronan Collobert, Jason Weston,Léon Bottou, Michael Karlen, Koray Kavukcuoglu, and Pavel Kuksa. Natural language processing (almost) from scratch. Journal of Machine Learning Research, 12(Aug):2493–2537, 2011.
\bibitem{Mikolov_2013} Tomas Mikolov, Ilya Sutskever, Kai Chen, Greg S Corrado, and Jeff Dean. Distributed representations of words and phrases and their compositionality. In Advances in neural information processing systems, pages 3111–3119, 2013.

\bibitem{Pennington_2014} Jeffrey Pennington, Richard Socher, and Christopher D Manning. Glove: Global vectors for word representation. In EMNLP, volume 14, pages 1532–1543, 2014.

\bibitem{Kim_2015} Yoon Kim, Yacine Jernite, David Sontag, and Alexander M Rush. Character-aware neural language models. arXiv preprint arXiv:1508.06615, 2015.
\bibitem{Chiu_2015} Jason PC Chiu and Eric Nichols. Named entity recognition with bidirectional lstm-cnns. arXiv preprint arXiv:1511.08308, 2015.
\bibitem{ijcai_2017} Jinhyuk Lee and Hyunjae Kim and Miyoung Ko and Donghee Choi and Jaehoon Choi and Jaewoo Kang, Name Nationality Classification with Recurrent Neural Networks, Proceedings of the Twenty-Sixth International Joint Conference on artificial Intelligence, {IJCAI-17}, 2081--2087, 2017, 10.24963/ijcai.2017/289, https://doi.org/10.24963/ijcai.2017/289,

\bibitem{Potdar_2017}K. Potdar, T. S., and C. D., “A Comparative Study of Categorical Variable Encoding Techniques for Neural Network Classifiers,”International Journal of Computer Applications, vol. 175, no. 4, pp. 7–9, 2017. [Online]. Available: https://doi.org/10.5120/ijca2017915495
\bibitem{Agirrezabal_2017} M. Agirrezabal, I. Alegria, and M. Hulden, “A Comparison of Feature-Based and Neural Scansion of Poetry,” Ranlp, 2017. [Online]. Available: https://arxiv.org/pdf/1711.00938.pdf
  
\end{thebibliography}


%\newpage
%\begin{subappendices}
%
%\section{Appendix A}\label{A}
%Please refer to Appendix \ref{C}.
%
%\section{Second appendix}\label{B}
%Please refer to Appendix \ref{A}.
%
%\section{Third appendix}\label{C}
%Please refer to Appendix \ref{B}.
%
%\end{subappendices}
\newpage
