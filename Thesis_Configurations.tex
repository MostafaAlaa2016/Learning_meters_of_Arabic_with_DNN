%------------------------------------------------------------
%
\documentclass[a4paper,12pt]{report}%%
%Options -- Point size:  10pt (default), 11pt, 12pt
%        -- Paper size:  letterpaper (default), a4paper, a5paper, b5paper
%                        legalpaper, executivepaper
%        -- Orientation  (portrait is the default)
%                        landscape
%        -- Print size:  oneside (default), twoside
%        -- Quality      final(default), draft
%        -- Title page   notitlepage, titlepage(default)
%        -- Columns      onecolumn(default), twocolumn
%        -- Equation numbering (equation numbers on the right is the default)
%                        leqno
%        -- Displayed equations (centered is the default)
%                        fleqn (equations start at the same distance from the right side)
%        -- Open bibliography style (closed is the default)
%                        openbib
% For instance the command
%           \documentclass[a4paper,12pt,leqno]{article}
% ensures that the paper size is a4, the fonts are typeset at the size 12p
% and the equation numbers are on the left side
%
%@@@https://codeinthehole.com/guides/writing-a-thesis-in-latex/


\usepackage{fontspec}
\usepackage{amsmath}
\newcommand*{\vertbar}{\rule[-1ex]{0.5pt}{2.5ex}}
\newcommand*{\horzbar}{\rule[.5ex]{2.5ex}{0.5pt}}
%\usepackage{geometry}
\usepackage[a4paper]{geometry}
\geometry{ a4paper, total={170mm,257mm}, left=12mm,right=20mm, top=25mm,}
%\usepackage[left=1in, right=1in, top=1in, bottom=1in, includefoot, headheight=5.6pt]{geometry}

\usepackage{makecell}

\setlength{\parskip}{6pt}
\usepackage{amsfonts}%
\usepackage{amssymb}%
\usepackage{graphicx}
%-------------------------------------------
\newtheorem{theorem}{Theorem}
\newtheorem{acknowledgement}[theorem]{Acknowledgement}
\newtheorem{algorithm}[theorem]{Algorithm}
\newtheorem{axiom}[theorem]{Axiom}
\newtheorem{case}[theorem]{Case}
\newtheorem{claim}[theorem]{Claim}
\newtheorem{conclusion}[theorem]{Conclusion}
\newtheorem{condition}[theorem]{Condition}
\newtheorem{conjecture}[theorem]{Conjecture}
\newtheorem{corollary}[theorem]{Corollary}
\newtheorem{criterion}[theorem]{Criterion}
\newtheorem{definition}[theorem]{Definition}
\newtheorem{example}[theorem]{Example}
\newtheorem{exercise}[theorem]{Exercise}
\newtheorem{lemma}[theorem]{Lemma}
\newtheorem{notation}[theorem]{Notation}
\newtheorem{problem}[theorem]{Problem}
\newtheorem{proposition}[theorem]{Proposition}
\newtheorem{remark}[theorem]{Remark}
\newtheorem{solution}[theorem]{Solution}
\newtheorem{summary}[theorem]{Summary}
\newenvironment{proof}[1][Proof]{\textbf{#1.} }{\ \rule{0.5em}{0.5em}}

%------------All done to create a subsubsubsection command with numbering-----------
\usepackage[T1]{fontenc}
%\usepackage[latin1]{inputenc}

\usepackage{lmodern}
\usepackage[english]{babel}
\usepackage{hyperref}
\hypersetup{
  colorlinks,
  citecolor=blue,
  filecolor=blue,
  linkcolor=blue,
  urlcolor=blue}
\setcounter{secnumdepth}{3}
% -------------------------------------------------------------
\setcounter{tocdepth}{2} %depth of table of contents
% \setcounter{secnumdepth}{-1}


\usepackage{setspace}
\usepackage{pslatex,palatino,avant,graphicx,color}
\usepackage{titlesec}
\titleformat{\chapter}[display]
{\normalfont\huge\bfseries\centering}{\thechapter}{20pt}{\Huge}
\newcommand{\cchapter}{\chapter}
\titleformat{\cchapter}[display]
{\normalfont\tiny\bfseries\centering}{\thecchapter}{12pt}{\tiny}
\usepackage{appendix}
\usepackage{float}
%%%%%%%%%%%%%%%%%%%%%%%%%%%%%%%%%%%%%%%%%%%%%%%%%%%%%%%%%%%%%%
% make the tables and the figures centered automatically
\makeatletter
\g@addto@macro\@floatboxreset\centering
\makeatother
%%%%%%%%%%%%%%%%%%%%%%%%%%%%%%%%%%%%%%%%%%%%%%%%%%%%%%%%%%%%%%

\usepackage[utf8x]{inputenc}
%\usepackage[utf8]{inputenc}
\usepackage[numbers,sort&compress]{natbib}
\usepackage{geometry}
%\usepackage{subcaption}
%\usepackage{subfig}
\usepackage{subfigure}

\usepackage{fourier}
\DeclareMathAlphabet{\mathcal}{OT1}{pzc}{m}{it}
\DeclareSymbolFont{letters}{OML}{cmm}{m}{it}
\usepackage{mathtools}
\DeclarePairedDelimiter{\ceil}{\lceil}{\rceil}
\usepackage{amsfonts, amssymb, bm}
\usepackage{booktabs}  % for \toprule ...
\usepackage{caption}
\usepackage{verbatim}
\usepackage{enumitem}
\newcommand{\dd}[1]{\mathrm{d}#1}

\usepackage{graphicx, type1cm, lettrine}
\usepackage{qtree}
\usepackage[dvipsnames, table]{xcolor}
\usepackage{collcell}
\usepackage{pgf}
\usepackage{forest}
\usepackage{setspace}
\usepackage{listings}

\usepackage{pgfplots}
%\usepackage[most]{tcolorbox}
% \tcbset{colback=yellow!10!white, colframe=red!50!black, 
%         highlight math style= {enhanced, %<-- needed for the ’remember’ options
%             colframe=red,colback=red!10!white,boxsep=0pt}
%         }
\usepackage{tikz}
\usetikzlibrary{shapes.geometric}
\usetikzlibrary{arrows.meta,arrows}
\usetikzlibrary{calc}

\usepackage{bidipoem}
\usepackage{polyglossia} % <3
\setmainlanguage{english}
\setotherlanguage{arabic}
\newfontfamily\arabicfont  [Script=Arabic, Scale=1.5]{Scheherazade}
\usepackage[numbers,sort&compress]{natbib}
\usepackage{float}
\usepackage{footnote}
\usepackage{pgfplots}
\usepackage{pgfplotstable}% loads pgfplots, tikz, graphicx
\usetikzlibrary{patterns}
\pgfplotsset{compat=newest}
%\pgfplotsset{every axis/.append style={tick label style={/pgf/number format/fixed},font=\scriptsize,ylabel near ticks,xlabel near ticks,grid=major}}

\usetikzlibrary{matrix}
\pgfplotsset{compat=1.14}% <- added!

\pgfplotsset{compat=1.8}
\usepackage{color}
 
\definecolor{codegreen}{rgb}{0,0.6,0}
\definecolor{codegray}{rgb}{0.5,0.5,0.5}
\definecolor{codepurple}{rgb}{0.58,0,0.82}
\definecolor{backcolour}{rgb}{0.95,0.95,0.92}
\usetikzlibrary{positioning}% To get more advances positioning options
%\usepackage{xcolor}

\colorlet{LightGreen}{green!25}
\colorlet{LightBlue}{blue!40}
\colorlet{LightPurple}{red!20}
\colorlet{LightYellow}{yellow!30}


\lstdefinestyle{mystyle}{
    backgroundcolor=\color{backcolour},   
    commentstyle=\color{codegreen},
    keywordstyle=\color{magenta},
    numberstyle=\tiny\color{codegray},
    stringstyle=\color{codepurple},
    basicstyle=\footnotesize,
    breakatwhitespace=false,         
    breaklines=true,                 
    captionpos=b,                    
    keepspaces=true,                 
    numbers=left,                    
    numbersep=5pt,                  
    showspaces=false,                
    showstringspaces=false,
    showtabs=false,                  
    tabsize=2
}
 
\lstset{style=mystyle}
%%%%%%%%%%%%%%%%%%%%%%%%%%%%%%%%%%%%%%%%%%%%%%%%%%%%%%%%%%%%
%%%%% highlight undefined reference in PDF


%\makeatother


%%%%%%%%%%%%%%%%%%%%%%%%%%%%%%%%%%%%%%%%%%%%%%%%%%%%%%%%%%%%
